% Literature review: 

% Local adaptation in the face of gene flow

% Richardson et al. 2014 present a review of cases of microgeographic evolution. Theory that selection must be strong to prevent homogeneization by gene flow has been developed by Felsenstein 1976, Garcia-Ramos and Kirkpatrick 1997, and Hendry et al. 2007. Empirical evidence for the counteracting effects of gene flow and selection has been found by Nosil and Crespi 2004 in stick insects, and Hendry et al. 2007 in sticklebacks. Willi and Hoffmann 2011 document micro-scale adaptation in Austrn fruit flies where populations from small f. Richardson and Urban 2013 same for salamanders

% The dewlap of Anolis lizards

% Dewlap and communication

% The anole dewlap is an important communication signal and anoles mainly use visual cues to communicate (Fleishman 1992). The dewlaps of most species reflect light between 300 (UV) and 700 (red)nm (Macedonia 2000, Nicholson et al. 2007). Most species have a very conserved tetrachromatic visual system that allows them to distinguish colors within this range (Fleishman 1993, 1997, Loew et al. 2002). Anoles have a good visual acuity (Fleishman 2001 cristatellus -- equation for detectability, 2017 sagrei). 

% Dewlap and sexual selection

% The anoline dewlap is sexually dimorphic in most species and has been hypothesized to be important for courtship and sexual selection in males. First, the dewlap may be a cue for female mate choice. Sigmund (1983) showed that females A. carolinensis prefer males with normal red dewlaps than painted green ones, and are sensitive to contrasts. However, Tokarz 2002 and Tokarz et al. 2005 found no evidence that mating success depends on dewlap extension in A. sagrei, although Driessens et al. (2014) found increased rates of dewlap display during courtship, together with pushups and headbobs. Although Vanhooydonck et al. (2009) did find a correlation between sexual size dimorphism and dewlap size, Baeckens et al. (2018) found no correlation between dewlap design and sexual size dimorphism (as proxy for sexual selection) across the West Indies in A. sagrei.

% Dewlap and male-male competition

% Second, the dewlap may be used in male-male competition. Vanhooydonck et al. (2005) found that males with relatively larger dewlaps have a stronger bite force in A. sagrei. Lower relative UV and orange-red coloration at the margin of the dewlap, together with display frequency, predict winners in male-male contests (Steffen and Guyer 2014).

% Dewlap as a quality indicator

% Third, the dewlap may convey information about the condition of the male, which may affect male-male competition as well as female mate choice. Cook et al. (2013) found duller dewlaps and fewer displays in heavily parasitized A. brevirostris, suggesting a trade-off between carotenoid use in the immune response and in color deposition. Driessens et al. (2015) found that general condition is best predicted by coloration, as more red and yellow dewlaps (relative to UV) correspond to better body condition and brighter dewlaps predict poorer immune response, and that dewlap use is not an indicator of quality. The edge of the dewlap conveys redundant information with dewlap center, except for yellow and UV color in this region that reflects hematocrit, another aspect of health status. 

% The female dewlap

% Female dewlaps may evolve relatively independently from male dewlaps. First, evidence for genetic correlation in dewlap features is limited. Cox et al. (2017) showed that hue covaries between the sexes, but brightness and relative area are genetically independent. Second, the selection pressures are different on male and female dewlaps. Harrison and Poe (2012) found that social selection (e.g. male mate choice, female-female competition) and sensory drive can explain variation female dewlap coloration across species, and found no support for genetic correlation and species recognition. Driessens et al. (2015) showed that females with larger or darker dewlaps display more to males, and Tokarz (2006) found that males indeed exert mate choice, preferring unfamiliar females.

% The dewlap against predators

% The dewlap can be used in interactions with predators, often together with other displays like pushup and headbob. Dewlap displayed are used in the presence of predators in A. cuvieri (Leal 1997) and A. cristatellus (Leal 1995), although for this species some predators like snakes, only pushups and headbobs increased in rate (Leal 1997) and push-ups are the main indicator of endurance for predator deterrence (Leal 1999). In A. sagrei, Driessens et al. 2014 found no increased dewlap use nor pushup or headbobs when interacting with predators. Although Vanhooyndonck et al. (2009) found relatively larger dewlaps in the presence of the predator Leiocephalus, Baeckens et al. 2018 found no link between predation pressure and dewlap design across islands of the West Indies. Predation may select against too conspicuous dewlaps, as Stuart-Fox et al. (2003) showed for the Australian rock dragons Ctenophorus, but not sure this applies to anoles, which can hide their dewlap.

% Secies recognition

% There is a large amount of evidence supporting a role of the dewlap in species recognition. Rand and Williams (1970) and Williams and Rand (1977) already suggested that dewlaps were more different between species in localities where there were more species on Hispaniola. Losos (1985) experimentally showed the role of species recognition between A. marcanoi and A. cybotes, which only differ in dewlap color. Fleishman (2000) reviewed the role for species recognition, also sensory drive and local adaptation. Leal and Fleishman (2002) found that A. cooki and A. cristatellus differ in coloration, which may be involved in species recognition. Macedonia and colleagues showed that A. grahami from Jamaic uses the dewlap as a species recognition signal, both against three sympatric species (1994) and against a robot (2013). Baeckens et al. (2018) recently found that spotted dewlaps appear more often on islands with more congeneric species of anoles in A. sagrei.

% Dewlap and speciation

% Ng et al. (2017) found that dewlap coloration acts as a species barrier between subspecies of A. distichus on Hispaniola. Lambert et al. (2013) found that dewlap coloration was under reproductive character displacement to increase species recognition in localities where subspecies overlap. In this system the subspecies have different dewlap colors and are genetically differentiated (Ng and Glor 2011). Geneva et al. (2015) showed that dewlap color differentiation correlates with speciation events in the distichus species group across Hispaniola and the Bahamas more generally. Nicholson et al. (2007) found no support for a species recognition role at the scale of the whole phylogeny. Ingram et al. (2016) found no link between dewlap size and speciation rates across the phylogeny, but found a speciational mode of evolution for this trait, suggesting it is under disruptive selection, but mainly for mainland lineages.

% Dewlap and local adaptation

% Dewlap design can correlate with aspects of the habitat, which suggests selection from the environment and a response through either local adaptation or plasticity. Macedonia (2001) described the rapid evolution of the blue dewlap of A. conspersus from Grand Cayman from its red dewlap relative from Jamaica,  A. grahami, in relation to changes in the environment and geology on this small island, and how those could explain the maintenance of body color polymorphism. Thorpe and Stenson (2002) found UV-richer dewlaps in the damper habitats of the Atlantic coasts of Martinique in A. roquet (where body coloration depends on altitude and humidity), and of St Vincent in A. trinitatis (Thorpe 2002). Leal and Fleishman (2002) found that the dewlaps of A. cooki and A. cristatellus on Puerto Rico each contrast more with their own background in UV light, suggesting local adaptation. Based on equations used to predict detectability (Fleishman 2001), they also showed that populations of A. cristatellus from different habitats (mesic or xeric), differed in dewlap coloration such that their dewlaps were optimally detectable in their surroundings, in line with the sensory drive hypothesis (Leal and Fleishman 2004). In a later study, however, Fleishman (2009) found that the dewlaps of four Purteo Rican species were not more detectable in their own habitat than in other species' habitats. Ng et al. (2012) showed that differences in dewlap coloration between subspecies of A. distichus on Hispaniola corresponded in differences in habitat, although the relationship between coloration and habitat-type were opposite to Leal and Fleishman (2002, 2004) where UV-contrast tended to be minimized. Those differences were maintained in the face of gene flow (Ng et al. 2016). Driessens et al. (2017) found significant correlations between dewlap characteristics and climatic variables across the West Indies in A. sagrei. Harrison and Poe (2012) found that differences in female dewlap coloration between species could be partly explained by differences in habitat-type. 

% Maintenance of dewlap differences despite gene flow

% Differences in dewlap can be maintained in the face of gene flow, suggesting they may be adaptive. This is the case for differences in color between subspecies of A. distichus on Hispaniola (Ng et al. 2016) but also in A. apletophallus from Panama (Stapley 2011) where populations vary in their frequencies of color morphs, without any detectable genetic differentiation. Finally, different body color morphs of A. conspersus coexist on the very small island Grand Cayman (Macedonia 2001). It is unknown whether it is also the case of A. sagrei, which was recently shown to be promiscuous (Kamath 2018), which would suggest high levels of gene flow on small islands such as the Bahamas and the Cayman Islands.

% Dewlap color production mechanism

% Dewlap coloration can be structural or pigmentary. The red-orange dewlap of anoles is made of (droso)pterins (Ortiz 1962, 1962a, Ortiz 1963, Ortiz 1966), which can be synthetized from guanine (Hill2006) and xantophyll carotenoids (Macedonia 2000) which must be found in the diet (Goodwin1984). They are present in similar concentrations across the whole surface of the dewlap in A. sagrei, while they differ in concentrations in differently colored areas in A. humilis (Steffen 2007). Sexes differ in their relative concentrations of both pigments and pigment concentration only partially explains differences in coloration (Steffen2009). Steffen (2010) showed that, although carotenoids cannot be synthetized by animals and must be found in their diet, coloration does not change in A. sagrei following a reduction or a supplementation of carotenoids in the food, contrary to what happens in birds (Hill 2002 finches, Hill 2006 textbook for all birds). Ng et al. (2013) also showed high heritability and little developmental plasticity of the orange coloration of males A. distichus facing the same kind of treatment. Cox et al. (2017) found high heritability of dewlap coloration in A. sagrei. However, one cannot rule out plasticity as a mechanism to explain the observed differences in coloration because most common garden experiments were done spanning one generation, and we now know that plasticity can be transmitted through multiple generations (Tariel et al. 2020).

% Nicholson et al. (2007) show that dewlap diversity does not match ecomorphs
