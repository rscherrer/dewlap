Animal signals evolve in an ecological context. Moreover, locally adapting animal sexual signals can be especially important for initiating or reinforcing reproductive isolation during the early stages of speciation. Dewlap color in \textit{Anolis} lizards can be highly variable between populations in relation to both biotic and abiotic adaptive drivers, albeit at relatively large geographical scales. At smaller scales, gene flow is likely to erase phenotypic differentiation between localities. Here, we investigated local adaptation of the dewlap across habitat-types at a small spatial scale, as this may give an indication of how conditions for the early stages of speciation may be met. We explored variation in dewlap coloration in the most widespread species of anole, \textit{Anolis sagrei}, across three characteristic habitats spanning the Bahamas and the Cayman Islands. Using reflectance spectrometry as well as supervised machine learning, we found significant differences in spectral properties of the dewlap between habitats within small islands, sometimes over very short distances. Passive divergence in dewlap phenotype associated with isolation-by-distance did not explain our results. On the other hand, these habitat-specific dewlap differences varied in magnitude and direction across islands, and thus our primary test for adaptation -- parallel responses across islands -- was falsified. We suggest, however, that selection could be involved and present several ways this might have happened, including sexual selection. (This one may be too much of a leap) While at present, populations from these different habitats probably experience too much gene flow to follow distinct evolutionary lineages, should additional barriers arise between habitat-specific populations, the observed disruptive selection on dewlap coloration may facilitate speciation.

