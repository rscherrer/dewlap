Animal signals evolve in an ecological context. Moreover, locally adapting animal sexual signals can be especially important for initiating or reinforcing reproductive isolation during the early stages of speciation. Dewlap color in \textit{Anolis} lizards can be highly variable between populations in relation to both biotic and abiotic adaptive drivers, albeit at relatively large geographical scales. At smaller scales, gene flow is likely to erase phenotypic differentiation between localities. Here, we investigated differentiation of dewlap coloration among habitat-types at a small spatial scale, within multiple islands of the West Indies, as this may give new insights into the local scale at which adaptation is possible. We explored variation in dewlap coloration in the most widespread species of anole, \textit{Anolis sagrei}, across three characteristic habitats spanning the Bahamas and the Cayman Islands. Using reflectance spectrometry as well as supervised machine learning, we found significant differences in spectral properties of the dewlap between habitats within small islands, sometimes over very short distances. Passive divergence in dewlap phenotype associated with isolation-by-distance did not explain our results. On the other hand, these habitat-specific dewlap differences varied in magnitude and direction across islands, and thus our primary test for adaptation -- parallel responses across islands -- was falsified. We suggest, however, that selection could be involved in several ways, including sexual selection. Our results shed new light on the scale at which signal color polymorphism can be maintained in the presence of gene flow, and the relative role of local adaptation and other processes in driving these patterns.

