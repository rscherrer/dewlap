 % Little but detectable differences on five islands

We tested for variation in \textit{A. sagrei} dewlap coloration between populations living in three characteristic habitat types across nine islands that span the West Indian range of the brown anole (Fig \ref{fig:map}, \ref{supfig:map}). We found that most of the variation in coloration is partitioned between islands (two-way semi-parametric MANOVA, modified ANOVA-type statistic (MATS) = 2009.6, P < 0.001, Fig. \ref{supfig:reflectance}, explained variance $\eta^2 = 44.3$\%, MANOVA approximation). Nonetheless, we did find evidence for differences in dewlap coloration between habitat-types, and those were mostly island-specific (habitat-by-island interaction term, MATS = 384.4, P < 0.001, explained variance $\eta^2 = 11.4$\%), leaving a small but significant portion of the variation explained by an archipelago-wide habitat effect (MATS = 42.5, P = 0.001, $\eta^2 = 4.8$\%).\\

The small archipelago-wide effect of habitat-type was detected for PC1, PC2 and PC3 (mixed-effect ANOVA with island as a random effect, Table \ref{suptab:anova-pooled}), but this effect was too small for \textit{post hoc} tests to find which habitats differed. Archipelago-wide differences in dewlap coloration between habitats were also detected by SVMs trained on pooled data regardless of island identity, both for PCA data and reflectance scores (Fig. \ref{supfig:classif-svm-pca-pooled}, \ref{supfig:classif-svm-refl-pooled}). This pattern seemed to be driven by mangrove lizards being correctly reassigned more often than predicted by chance. Sensitivity analyses on these machines suggest a relatively small role of long wavelengths (red reflectance) in driving this pattern (Fig. \ref{supfig:importance-svm-refl-pooled}), but did not reveal strong differences between the PCs in relative importance (Fig. \ref{supfig:importance-svm-pca-pooled}).  Archipelago-wide differences were not detected by LDA classifiers at all (Fig. \ref{supfig:classif-lda-pca-pooled}, \ref{supfig:classif-lda-refl-pooled}).\\ %So, the differences could be in something other than means.\\

Within islands, SVM classifiers correctly assigned individuals to their habitat of origin based solely upon dewlap coloration on five islands: Abaco, Bimini, Cayman Brac, Little Cayman, and Long island (Fig. \ref{fig:classif-svm-pca}). An LDA approach yielded similar success rates (Fig. \ref{supfig:classif-lda-pca}), suggesting robust differences between these populations. Of the five islands, Little Cayman was the best discriminated with a mean SVM generalization success of 73.4\% (Table \ref{suptab:classif-svm-pca}). The results of the classification analyses on PCA data were very similar to results from SVMs and LDAs trained on reflectance values at 50nm-spaced wavelengths from 300 to 700nm (Fig. \ref{supfig:classif-svm-refl} and \ref{supfig:classif-lda-refl}).\\

Differentiation in dewlap coloration occurred in multiple dimensions of color space. Moreover, the differences in dewlaps between habitats were not always consistent among islands, thus, we will discuss the habitat-specific variation in dewlap coloration for each island where significant differences were detected in turn (Fig. \ref{fig:anova}, Tables \ref{tab:anova}, \ref{suptab:kruskal}). Figure \ref{fig:anova}A provides a key to map principal component scores to the underlying wavelengths.\\

On Abaco, dewlaps did not differ in PC1, which represents brightness. Mangrove lizards had significantly lower PC2 scores, corresponding to higher ultraviolet reflectance and lower red reflectance. Coastal beach scrub lizards had lower scores on PC3, corresponding to lower ultraviolet reflectance and higher blue reflectance.\\

On Bimini, coastal beach scrub lizards had significantly brighter dewlaps than lizards from mangroves (PC1), but mangrove lizards had higher PC2 scores than beach scrub lizards, indicating higher violet and blue reflectance, and lower red reflectance. Lizards from primary coppice had higher PC3 scores overall, which correlated very positively with ultraviolet reflectance.\\

On Cayman Brac, coppice-lizard dewlaps were significantly less bright than lizards from the other habitats. Coastal beach scrub lizards had dewlaps that scored low on PC2, corresponding to lower violet-blue and more red, while the mangrove lizards exhibited the opposite: relatively higher levels of violet-blue and less red. In PC3 space we found that dewlaps from lizards in the coastal habitat had high ultraviolet reflectance, coppice lizards had intermediate levels, and mangrove lizards had relatively low levels.\\

On Little Cayman, the dewlaps of coppice lizards were significantly darker (PC1) than coastal-lizards. Mangrove lizards had less ultraviolet and redder dewlaps (PC2). The dewlaps of the coastal beach scrub lizards had higher levels of red and ultraviolet reflectance and less blue reflectance than the dewlaps of the other habitat-populations (PC3).\\

On Long Island, lizards from the coppice habitat had darker dewlaps than lizards from the other habitats (PC1). Coastal lizards had relatively more ultraviolet and less blue-green reflectance in their dewlaps (PC3). These coastal-habitat lizards also scored lower on PC4, corresponding to slightly more violet and green-yellow dewlaps, and less blue dewlaps, than the mangrove lizards on the island.\\

Sensitivity analyses on classifiers suggested an overall higher relative importance for PC2 and PC3 in determining between-group differences on Abaco, both in SVM and LDA classifiers (Fig. \ref{supfig:importance-svm-pca}, \ref{supfig:importance-lda-pca}), consistent with our ANOVA results (Fig. \ref{fig:anova}B). There was no strong signal of differences in relative importance among principal components on the other islands. Sensitivity analyses of SVMs trained on reflectance scores rather than principal components revealed, however, a consistently higher importance of ultraviolet reflectance in between-group differences on all islands (Fig. \ref{supfig:importance-svm-refl}). This pattern was not recovered for LDAs trained on reflectance scores (Fig. \ref{supfig:importance-lda-refl}).\\

% No spatial autocorrelation

We did not find significant spatial autocorrelation between the sampling sites on the islands where we detected a significant habitat effect. We did, however, detect a significant positive signal of autocorrelation on Eleuthera (Table \ref{suptab:autocor}), suggesting possible color differentiation through isolation-by-distance on this island.\\

In contrast, differences in dewlap coloration between habitats were often detected in close geographical proximity. For example, Bimini, Cayman Brac, and Little Cayman were among the smallest islands in our study (Fig. \ref{supfig:map}). Indeed, most significant differences in dewlap coloration involved sites that were 5-10km apart. Our most extreme case of significant differences occurred for PC3 between a beach scrub site and a coppice site, separated from each other by a few hundreds of meters at most on Bimini (multiple pairwise Wilcoxon-Mann-Whitney tests, Fig. \ref{supfig:distances}).\\

% Patterns of differentiation were inconsistent across islands

Patterns of differentiation were inconsistent across the five most significant islands. Contrasts in principal components between habitats, calculated on pooled data from the whole archipelago, were not similar, for any component, among islands (Fig. \ref{supfig:contrasts}; MANOVA, Pillai's trace = 0.354, $F(6, 32) = 1$, P = 0.36). No pattern of variation was shared by all five significant islands, along any dimension. Some patterns did seem more common however, such as darker dewlaps among coppice lizards (Cayman Brac, Little Cayman, and Long Island, Fig. \ref{fig:anova}) or the intermediate position of coppice lizards in chromatic color space (Cayman Brac and Long Island). In other cases, patterns of differentiation were reversed from one island to another, with more ultraviolet reflecting dewlaps in mangroves than in coastal habitat on Abaco and Cayman Brac, but the opposite on Little Cayman and Long Island. Overall, it seemed that patterns of heterogeneity of variance were often driven by higher variances in coloration within beach scrub lizards (Fig. \ref{fig:anova}, Table \ref{tab:anova}). Yet other patterns were idiosyncratic, such as the combination of higher red and ultraviolet reflectance in coastal lizards on Little Cayman, where the rule seemed to be a negative correlation between ultraviolet and red reflectance across every other island.\\