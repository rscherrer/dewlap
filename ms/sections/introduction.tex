% Animal signals may be shaped by their environment

The staggering diversity of animal communication signals has long been of interest to evolutionary biologists. Animals use chemical, mechanical, electromagnetic, and visual signals to communicate in a wide variety of contexts, including, competition for mates, species recognition, aposematism, and cooperation \citep{Bradbury2011}. A primary evolutionary factor shaping communication signals is the sensory system and behavior of recipients (the sensory drive hypothesis; \citealt{Endler1988,Endler1992,Endler1998}). Over the past decades, scientists have established that signals evolve in an ecological context and are dependent on environmental conditions \citep{Endler1992,Endler1993,Endler1993a}. Just as different habitats may favor different combinations of ecomorphological traits to maximize performance and fitness \citep{Arnold1983}, they may also shape different forms of a signal, so as to maximize its transmission and detection (e.g. \citealt{Seehausen1997}), or reduce its detection by unintended recipients such as predators \citep{Endler1984,Endler1990,Endler1991,Halfwerk2014}. This selective pressure may drive the local adaptation of communication signals.\\

% One main barrier to the maintenance of diversity is gene flow

One potential barrier to the maintenance of localized signal divergence is the homogenizing effect of gene flow. Population genetics theory suggests that gene flow may counteract local adaptation between localities and prevent divergence altogether, especially at small spatial scales, because of the inflow of maladapted alleles or because of the breaking of linkage between coevolving loci \citep{Felsenstein1976, Garcia-Ramos1997, Dieckmann1999, Lenormand2002, Hendry2007}. This genetic homogenization has been confirmed empirically in systems such as stick insects \citep{Nosil2004} and stickleback \citep{Hendry2007a}. Yet, examples of microgeographic adaptation, i.e. adaptation at smaller scales than the range of dispersal, exist, highlighting the potential of some organisms to respond to selection in the face of gene flow (see \citealt{Richardson2014} and references therein). Examples include small scale adaptation in fragmented areas in Australian fruit flies \citep{Willi2012}, and local adaptation to predation pressure in North American salamanders \citep{Richardson2013}. Therefore, despite evidence that local adaptation may be particularly difficult at small spatial scales where gene flow tends to cause adjoining populations to remain genetically homogeneous, the potential adaptive response of species traits, in particular communication signals, to localized differences in habitats remains relatively unknown \citep{Richardson2014}. Lizards of the neotropical genus \textit{Anolis} are an excellent group for studying the eco-evolutionary dynamics of local adaptation and natural selection \citep{Losos2009}. A particularly conspicuous trait of anoles is their dewlap, an extensible flap of skin that is typically sexually dimorphic and used as a communication signal in courtship \citep{Sigmund1983, Driessens2014, Driessens2015} and territorial displays \citep{Losos1985, Macedonia1994, Macedonia2013} as well as in predator deterrence \citep{Leal1995, Leal1997, Leal1997a}. Dewlap characteristics vary widely among the approximately $400$ species of the genus \citep{Nicholson2007}. Interspecific variation in dewlap coloration is implicated in species recognition \citep{Rand1970, Williams1969, Williams1977, Losos1985, Macedonia1994, Fleishman2000, Macedonia2013}, and this function could have had a role in initiating or reinforcing reproductive isolation during speciation \citep{Lambert2013, Geneva2015, Ng2017}.\\

Within species, studies have shown a link between variation in dewlap coloration and differences in habitats or climatic conditions \citep{Macedonia2001, Leal2002, Thorpe2002, Thorpe2002a, Leal2004, Vanhooydonck2009, Ng2012, Ng2013, Ng2016, Vanhooydonck2009, Driessens2017}. Some studies suggest that those differences may be adaptive and that dewlaps may have evolved to maximize detectability given local light conditions \citep{Fleishman2001, Leal2002, Leal2004}. Although this claim is further supported by recent findings that dewlap colors are perceived differently under different levels of shading \citep{Fleishman2020}, other studies found conflicting patterns of between-habitat variation that did not support the sensory drive hypothesis \citep{Fleishman2009, Ng2012, Macedonia2014}.\\ 

Previous studies investigating variation in anole dewlaps compared populations at relatively large geographical scales, e.g. between islands \citep{Vanhooydonck2009, Driessens2017} or within large islands such as Puerto Rico \citep{Leal2004} or Hispaniola \citep{Ng2012, Ng2016}. These large scales and marine barriers should reduce gene flow \citep{Ng2011, Lambert2013, Richardson2014, Ng2017}. That said, examples do exist of divergence in dewlap coloration at smaller scales or between populations with high degrees of gene flow \citep{Thorpe2002, Thorpe2002a, Stapley2011, Ng2016}.\\

The species \textit{Anolis sagrei} is widespread across islands of the West Indies \citep{Reynolds2020}. It has been the subject of numerous studies concerning local adaptation \citep{Losos1994, Losos1997a, Losos2001, Kolbe2012}, biological invasion \citep{Kolbe2008}, and sexual selection \citep{Tokarz2002, Tokarz2005, Tokarz2006, Driessens2014, Steffen2014, Driessens2015} among many other topics. Between-island variation in the mainly orange-red color of its dewlap was shown to be better explained by climatic variables \citep{Driessens2017} than by proxies for biotic factors such as sexual selection or predation pressure \citep{Vanhooydonck2009, Baeckens2018}. How intra-island differences in habitat may contribute to the diversity of dewlap coloration, however, remains unexplored, and may reveal new insights into the scale of local differentiation despite gene flow.\\

% What we did

Here, we analyzed the color characteristics of \textit{A. sagrei} dewlaps within nine islands in the Bahamas and Cayman Islands. These island systems presently, if not historically, comprise relatively small islands, with no major geographic barriers within islands limiting dispersal for this promiscuous species \citep{Kamath2018}. These islands all share three characteristic native West Indian small-island habitat-types -- beach scrub bush, closed-canopy primary coppice forest, and mangrove forest -- that are often spatially intermingled. These habitats contrast in environmental parameters including vegetation community, light irradiance, humidity, and temperature \citep{Howard1950, Schoener1968}. The Cayman Islands and the Bahamas have been colonized independently by \textit{A. sagrei} from Cuba (\citealt{vandeSchoot2016} unpublished thesis; \citealt{Reynolds2020}), such that these archipelagos constitute an ideal suite of natural replicates to explore within-island dewlap diversity across multiple islands.\\

Our sampling design included sites in close proximity; the median distance between two sites within an island was $11.2$km. Combining reflectance spectrometry and supervised machine learning, we tested for divergence in dewlap phenotype between habitats within islands and between islands across part of the range of \textit{A. sagrei}. We predicted that if light conditions in the environment indeed drive color evolution, dewlaps should be most similar between beach scrub and mangrove forest, which both have high levels of light irradiance, compared to the darker, closed-canopy coppice forest. Similar, if detectability is maximized given the local conditions, we expected darker and more contrasting dewlaps in high irradiance habitats. Finally, if habitat characteristics are strong determinants of dewlap color variation, similar patterns should be observed across multiple islands \citep{Losos2011}. 

% Jonathan does not like when we sum up the results at the end of the intro

% We found strong support for fine-scale, within-island differences in coloration between lizards inhabiting the three habitat-types in several color space dimensions, suggesting a potentially strong effect of divergent selection. However, the divergence patterns we observed did not match our \textit{a priori} predictions and were inconsistent between islands. We found no evidence of isolation-by-distance as an explanation for the observed differences. Our results are nevertheless consistent with small-scale adaptive maintenance of signal polymorphism despite presumed considerable opportunity for gene flow, and might suggest that idiosyncracies of local drift, selection, and gene flow contribute to differing outcomes in different populations.

% Bibliography notes:

% Anolis sagrei was shown to be able to rapidly invade available ecological niches upon colonization of new islands through changes in limb proportions (Losos1997a; Losos2001)

% Convergence

% Besides, \textit{Anolis} lizards are well-known examples of eco-morphological evolutionary convergence, where multiple ``ecomorphs'' adapted independently, but in similar ways, to similar ecological niches across islands of the Greater Antilles (Losos2004; Losos2009). This extent of evolutionary determinism seems to act within species at present too. Populations of \textit{A. sagrei} were shown to undergo the same eco-morphological divergence upon colonization of new islands as ecomorphs did millions of years before (Losos1997a; Losos2001; Calsbeek2007; Losos2007). Strong developmental constraints on morphology seem to explain this tendency (Sanger2012; McGlothlin2018). However, despite the evidence suggesting potential for evolutionary convergence at the intra-specific level in anoles, this pattern is unknown for communication signals such as dewlap coloration. Populations of a same species, because they share a common genetic architecture, are expected to evolve convergent adaptations more often than distantly related species (Ord2015), and this could apply to the dewlap of \textit{A. sagrei}. On the other hand, sexual signals are often evolutionarily labile due to sexual selection (Kraaijeveld2011), but also selection mediated by other recipients of the signal (Endler1981; Endler1988; Endler1984; Endler1993; Endler1993a; Endler1998). This is consistent with the high diversity of dewlaps across anole species (Nicholson2007) and could imply evolutionary contingency for this colorful trait.\\
