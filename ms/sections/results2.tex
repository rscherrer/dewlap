% Little but detectable differences on five islands

We tested for variation in \textit{A. sagrei} dewlap coloration between populations living in three characteristic habitat types across nine islands that span the West Indian range of the brown anole (beach scrub, primary coppice and mangroves). We found that most of the variation in coloration was partitioned between islands (two-way PERMANOVA, $F(df = 8) = 43.7$, $P = 0.0001$, explained variance $R^2 = 40.9$\%). Nonetheless, we did find evidence for differences in dewlap coloration between habitat types, and those were mostly island-specific (habitat-by-island interaction term, $F(16) = 3.53$, $P = 0.0001$, $R^2 = 6.6$\%), with a significant portion of the variation explained by an habitat effect across all islands, but this effect was relatively small ($F(2) = 4.7$, $P = 0.0001$, $R^2 = 1.1$\%).\\

We subsequently tested for differences in dewlap coloration between habitat-populations within each island, using within-island principal component scores (to maximize the variation captured for each island, see Methods). Our within-island random forest classification analyses revealed detectable differences in dewlap coloration on eight out of the nine islands in our sample: Abaco, Bimini, Cayman Brac, Eleuthera, Little Cayman, Long Island, North Andros and South Andros. The accuracy of random forest classification exceeded random expectation more often than expected by chance for all these islands (Table \ref{tab:randomforests}). Accuracy was as high as 73\% for Cayman Brac. We obtained similar results using other machine learning approaches such as support vector machines (Table \ref{tab:ksvms}) and linear discriminant analysis (Table \ref{tab:ldas}), except that these methods did not detect significant differences on Eleuthera and North Andros. We describe in details the specific differences detected on each island in the Appendix, and focus here on the general patterns emerging from our data.\\

Overall, we found significant differences in dewlap coloration between populations that were often in close geographical proximity. On Bimini, notably, we found a significant difference between dewlaps from beach scrub and primary coppice forest, at a distance of a few hundred meters, making this contrast the smallest geographical scale at which differences in coloration were found in our study (Fig. \ref{fig:Bimini}). We also detected significant differences in dewlap coloration at distances below one kilometer on Abaco (Fig. \ref{fig:Abaco_supplement}G), and at distances between one and ten kilometers on Bimini (Fig. \ref{fig:Bimini}G), Cayman Brac (Fig. \ref{fig:CaymanBrac}G), Little Cayman (Fig. \ref{fig:LittleCayman}G), Long Island (Fig. \ref{fig:LongIsland}G) and South Andros (Fig. \ref{fig:SouthAndros}G).\\

We found evidence of spatial autocorrelation in dewlap coloration between the sites within islands for Abaco (Table \ref{tab:autocorrelation}), suggesting that populations from closer sites tend to have more similar dewlaps on this island than expected by chance. Abaco was the island we sampled at the largest scale, with some sites nearly a hundred kilometers away from each other (Fig. \ref{fig:Abaco}A). That said, some sites were also in close proximity, and significant differences in coloration were detected between habitats sometimes less than a kilometer away (Fig. \ref{fig:Abaco_supplement}G), suggesting that differences in dewlap coloration between distant sites may be partly attributable to isolation-by-distance, but this may not necessarily be the case for sites in close proximity. We did not find evidence for spatial autocorrelation on other islands than Abaco (Table \ref{tab:autocorrelation}).\\

A striking feature of our data was inconsistencies in between-habitat differences among islands, in terms of which habitats differ from which, which dimensions of coloration were involved, and in which direction. For example, while on Cayman Brac the random forests could well distinguish between all three habitats (Fig. \ref{fig:CaymanBrac}D), on Abaco dewlaps from beach scrub and primary coppice were often mistaken, and on Bimini beach scrub dewlaps were more often classified into primary coppice or mangrove than into beach srub (Fig. \ref{fig:Bimini}D). In terms of variable importance, for multiple islands the random forests used information in the UV range to discriminate between at least some habitats, particularly on Abaco (Fig. \ref{fig:Abaco_supplement}F), Bimini (Fig. \ref{fig:Bimini}F), Cayman Brac (Fig. \ref{fig:CaymanBrac}F), Little Cayman (Fig. \ref{fig:LittleCayman}F) and Long Island (Fig. \ref{fig:LongIsland}F), but differences in UV reflectance involved different habitats and were in different directions among these islands.\\