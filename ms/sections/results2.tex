% Little but detectable differences on five islands

We tested for variation in \textit{A. sagrei} dewlap coloration between populations living in three characteristic habitat types across nine islands that span the West Indian range of the brown anole (coastal scrub, primary coppice and mangroves). We found that most of the variation in coloration was partitioned between islands (two-way PERMANOVA, approx. $F(\text{df} = 8) = 43.7$, $P = 0.0001$, Fig. \ref{fig:reflectance}, explained variance $R^2 = 40.9$\%). Nonetheless, we did find evidence for differences in dewlap coloration between habitat-types, and those were mostly island-specific (habitat-by-island interaction term, approx. $F(16) = 3.53$, $P = 0.0001$, $R^2 = 6.6$\%), with a small but significant portion of the variation explained by an archipelago-wide habitat effect (approx. $F(2) = 4.7$, $P = 0.0001$, $R^2 = 1.1$\%).\\

We subsequently tested for differences in dewlap coloration between habitat-populations within each island, using within-island principal component scores, that is, computed from the data specific to each island (to maximize the variation captured for each island). Our within-island random forest classification analyses revealed detectable differences in dewlap coloration on seven out of the nine islands in our sample: Abaco, Bimini, Cayman Brac, Eleuthera, Little Cayman, Long Island and South Andros. On these islands, our classifiers could reassign individual dewlaps to their correct habitat more often than expected by chance (Table \ref{tab:randomforests}). We obtained nearly identical results using support vector machines (Table \ref{tab:ksvms}) and linear discriminant analysis (Table \ref{tab:ldas}) for classification. We did not find evidence of spatial autocorrelation in dewlap coloration between the sites within islands, except for Eleuthera (Table \ref{tab:autocorrelation}). We now describe the specific differences detected on each island.\\

On Abaco, we detected the strongest differences between the coastal and mangrove habitats, while dewlaps from the coppice habitat were more difficult to classify (Fig. \ref{fig:Abaco}A). Importance analysis revealed that coastal and mangrove lizards mostly differed in reflectance in the UV-end of the spectrum (below 400nm, Fig. \ref{fig:Abaco}B), where mangrove dewlaps had higher UV reflectance relative to coastal lizards, and coppice lizards had an intermediate UV reflectance between the two other habitats (Fig. \ref{fig:Abaco}C). Consistent with this, our analyses of variance detected significantly lower PC2 scores in mangrove lizards than in the two other habitats, representing a higher UV-reflectance relative to red (Fig. \ref{fig:Abaco}D, E, Table \ref{tab:anova}). Coastal lizards also scored lower on PC3, indicating less curvature of the reflectance profile and relatively higher reflectance at intermediate wavelengths (blue-to-yellow) than at the ends of the range (Fig. \ref{fig:Abaco}D, E). Differences were detected between sites both at large ($\sim$ 100km) and short ($<$ 1km) distances (Fig. \ref{fig:Abaco}F, G).\\

On Bimini, the random forests mostly correctly classified lizards from the coastal habitat (Fig. \ref{fig:Bimini}A), with a relatively flat importance profile suggesting that brightness was used instead of a particular wavelength (Fig. \ref{fig:Bimini}B). Indeed, some coastal dewlaps were substantially brighter than the rest (Fig. \ref{fig:Bimini}C), a pattern that was captured by our analysis of variance along PC1 (i.e. brightness, Fig. \ref{fig:Bimini}D, E, Table \ref{tab:anova}). Coastal lizards were also characterized by elevated red reflectance relative to UV (as represented by PC2, (Fig. \ref{fig:Bimini}D, E)), and coastal and mangrove lizards were characterized by a more even distribution of the reflectance along the spectrum (as represented by PC3, (Fig. \ref{fig:Bimini}D, E)), in contrast to coppice lizards which harbored a stronger curvature at intermediate wavelengths (Fig. \ref{fig:Bimini}D, E). On this island, the coastal and coppice habitats were separated by a few hundred meters, making this contrast the smallest geographical scale at which differences in coloration were found in our study (Fig. \ref{fig:Bimini}F, G).\\

On Cayman Brac, all three habitats could be fairly well discriminated against each other (Fig. \ref{fig:CaymanBrac}A), with UV reflectance appearing to be again an important variable (Fig. \ref{fig:CaymanBrac}B). Coastal and mangrove lizards were the best differentiated habitats (Fig. \ref{fig:CaymanBrac}A). At a distance between 2 and 3km (Fig. \ref{fig:CaymanBrac}F, G), dewlaps in the coastal habitat reflected more red light (as represented by PC2, Fig. \ref{fig:CaymanBrac}D, E) and more UV (as represented by PC3) than in the mangrove habitat. Coppice lizards were characterized by darker dewlaps than the rest (represented by PC1), and a higher UV reflectance than mangrove dewlaps (PC3, Fig. \ref{fig:CaymanBrac}D, E, Table \ref{tab:anova}).\\

Eleuthera was the only island where we detected significant spatial autocorrelation (Table \ref{tab:autocorrelation}), that is, sites that were closer geographically tended to have populations of lizards with more similar dewlap colors. The strongest identified differences were between coastal and mangrove lizards (Fig. \ref{fig:Eleuthera}A), where coastal lizards had higher levels of red reflectance and mangrove lizards higher levels of UV reflectance (as represented by PC2, Fig. \ref{fig:Eleuthera}D, E, Table \ref{tab:anova}).\\

Little Cayman was also characterized by a good discrimination of all three habitats, particularly of mangrove lizards (Fig. \ref{fig:LittleCayman}A). These lizards differed in short wavelengths from the rest (Fig. \ref{fig:LittleCayman}B), with significantly lower UV reflectance (as represented by PC2, Fig. \ref{fig:LittleCayman}D, E, Table \ref{tab:anova}). Coastal lizards were characterized by brighter dewlaps than coppice lizards (PC1), and also more convex curves, i.e. slightly higher UV and red reflectance (as represented by higher PC3 scores), than lizards from the other two habitats (Fig. \ref{fig:LittleCayman}D, E, Table \ref{tab:anova}).\\

On Long Island the three habitats were well discriminated too, with the largest differences between coastal and mangrove lizards (Fig. \ref{fig:LongIsland}A). Coastal lizards had more curved reflectance profiles than in either of the two other habitats, with higher levels of UV and red reflectance relative to intermediate wavelengths (PC3, Fig. \ref{fig:LongIsland}D, E, Table \ref{tab:anova}). Coastal lizards also differed from mangrove lizards along PC4 (Fig. \ref{fig:LongIsland}D), which represents a rather small portion of the variance not already explained by the first three principal components, and is therefore difficult to interpret (Fig. \ref{fig:LongIsland}E). Coppice lizards were significantly darker than mangrove and coastal lizards (PC1, Fig. \ref{fig:LongIsland}D, E, Table \ref{tab:anova}).\\ 

On North Andros, although the random forest classification was significant ($P = 0.0216$, Table \ref{tab:randomforests}) and the average confusion matrix indicated that lizards from beach scrub were particularly well predicted (Fig. \ref{fig:NorthAndros}A), no significant univariate differences were detected along any of the four PCs (Fig. \ref{fig:NorthAndros}D, Table \ref{tab:anova}). Importance analysis of full-spectrum random forests showed higher importance scores near the UV-end of the spectrum in discriminating beach scrub dewlaps from the rest (Fig. \ref{fig:NorthAndros}B). Besides, reflectance curves of beach scrub dewlaps appeared more similar to each other in the UV range than dewlaps from other habitats (Fig. \ref{fig:NorthAndros}C), suggesting that the machines may have used this low within-habitat variance, as opposed to between-habitat differences in means, to correctly classify beach scrub lizards. A small sample size on this islands may also have contributed to a lack of power in detecting univariate differences using analyses of variance (Table \ref{tab:counts}).\\

On South Andros classification was not as successful as on other islands, but coastal and coppice dewlaps could be discriminated better than expected by chance (Fig. \ref{fig:SouthAndros}A). Coppice lizards had more curved reflectance profiles than coastal lizards (PC3), and lizards from both habitats differed along PC4, which is again more difficult to interpret (Fig. \ref{fig:SouthAndros}D, E, Table \ref{tab:anova}). Coastal lizards also differed from mangrove lizards in PC4 (Fig. \ref{fig:SouthAndros}D, E, Table \ref{tab:anova}).\\

Classification success was not significantly better than expected by chance on Ragged Island (Table \ref{tab:randomforests}) where nearly no habitat could be differentiated from any other based on reflectance.