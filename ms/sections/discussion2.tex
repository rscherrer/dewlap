Two main insights follow from our results. First, excluding North Andros where the follow-up univariate analyses were not significant, we detected significant differences in dewlap coloration between habitats within seven out of the nine islands investigated, suggesting a putatively high potential for local differentiation of dewlap coloration in \textit{Anolis sagrei}. Second, we found differences in coloration along different dimensions of color space, and in different directions, on different islands.\\

Detectable differences in dewlap color between habitat-populations are surprising, as habitats were often in close geographical proximity to each other (as close as a few hundred meters on Bimini and most of the time within ten kilometers). Indeed, given that (1) the populations were continuously distributed between the habitats, (2) different habitat-populations were not monophyletic with respect to mitochondrial haplotypes (van de Schoot, unpublished thesis), and (3) \textit{A. sagrei} have been shown to be a highly mobile species within these islands \citep{Kamath2018}, we would have expected more homogeneous distributions of color phenotypes within islands due to extensive gene flow, with fewer differences between populations, especially those in close proximity.\\

Several scenarios could account for these findings. One explanation is an adaptive one. Indeed, populations living in different habitats could be phenotypically adapted to their local environmental conditions. Given that the brightly colored dewlap of \textit{A. sagrei} is used as a communication signal, its color may be a target for selection if the transmission or perception of the signal differs from one habitat to another, for example because of differences in ambient light, according to the sensory-drive hypothesis \citep{Endler1988, Endler1992, Endler1998}. The sensory-drive hypothesis has been tested multiple times for dewlap coloration in \textit{Anolis} lizards, with mixed results. Some authors found support for it \citep{Leal2002, Leal2004}, while others did find differences in dewlap coloration between habitats, but those were inconsistent with the sensory-drive hypothesis \citep{Fleishman2009, Ng2012}.\\

If our results were an example of sensory drive, we would have expected to see consistent differences between habitat-populations across islands (a pattern that would have been a compelling indicator of adaptation at all, \citealt{Losos2011}). This is because environmental conditions that may be relevant to color signal detectability such as light, temperature, moisture and vegetation, are consistent within the three main and clearly distinct habitat-types found across the sampled islands, i.e. beach scrub, primary coppice and mangroves \citep{Howard1950, Schoener1968}. Moreover, the patterns of divergence expected under a sensory drive scenario should be consistent with increased detectability given the local light conditions, such as the high contrasts with background vegetation found in the UV-range by \citet{Leal2002} and \citet{Leal2004}.\\

Instead, we found differences in the way dewlap color differs between habitats across islands. While short-wavelengths (UV reflectance) were often involved in color differences, they were not involved on all islands where significant differences were detected. On some islands, other or additional variables differed, such as brightness, red reflectance or the reflectance at the ends of the spectrum visible to \textit{Anolis} lizards (UV and red, \citealt{Lazareva2012}) relative to intermediate wavelengths (blue-to-yellow). Similar portions of the spectrum were sometimes involved in opposite directions on different islands, such as on Abaco and Cayman Brac, where mangrove lizards had a higher UV-reflectance than beach scrub lizards on the former, but a lower UV-reflectance on the latter. Under a sensory-drive scenario, we would have expected the same variables to be consistently divergent between habitats, or at least in a consistent direction.\\

Not only consistent patterns across islands would have been a good clue for a sensory-drive explanation, but in particular consistent differences between habitats that are most different in their local conditions regarding the ecological function of the dewlap, such as ambient light. For example, if ambient light is an important factor shaping dewlap coloration, we would expect mangrove and beach scrub lizards, both inhabiting areas with high light penetration, to harbor more similar dewlaps, and to differ significantly from lizards from the coppice habitat, where irradiance is low. Overall, the observed heterogeneity of divergence patterns across islands provides no support to a sensory-drive explanation.\\

Phenotypic plasticity could be another cause for dewlap color variation between habitats, where different conditions would favor different phenotypes in different habitats, without genetic changes. Indeed, the yellow, orange and red colors in anoline dewlaps are produced by pterins and carotenoids \citep{Ortiz1962, Ortiz1962a, Ortiz1963, Ortiz1966, Macedonia2000, Steffen2007, Steffen2009}. Animals can be synthesize pterins from nucleotides, but lack the ability to synthesize carotenoids \citep{Goodwin1984, Hill2002, Hill2006}. Different food qualities across sites within islands could therefore potentially cause detectable differences in coloration. Alternatively, more subtle effects on dewlap color could arise from developmental plasticity and depend, e.g. on differences in egg-rearing conditions. However, more data are needed to test these hypotheses, and although some work has shown plastic responses of dewlap color in response to parasites in \textit{A. sagrei} \citep{Cook2013}, we find it unlikely to account for the widespread habitat differences we found. Besides, studies testing the effect of carotenoid deprivation \citep{Steffen2010, Ng2013} and heritability \citep{Cox2017} of dewlap coloration in \textit{A. sagrei} and another species with a carotenoid-based dewlap, \textit{A. distichus}, found little support for phenotypic and developmental plasticity in dewlap coloration.\\

Genetic drift is another process that can account for differences in phenotype between localities, especially in small populations. One way this could proceed is through isolation-by-distance, where more distant populations accumulate more differences through time because of the reduced effect of gene flow at larger geographical scales relative to the dispersal range of the species \citep{Rousset2004}. Here, we only found a significant correlation between phenotypic and geographical distances on Eleuthera to support this scenario. On all the other islands, in contrast, populations from closer sites were not phenotypically more similar, which argues against isolation-by-distance. That said, there were often few sampling sites per island in our study, whose locations were not uniformly chosen within the islands, and so the true extent of isolation-by-distance may be difficult to test. Other, less trivial forms of drift may be at play than isolation-by-distance, but nevertheless, we did find significant differences in color phenotype at relatively small spatial scales, sometimes in neighboring habitats, on islands where gene flow is probably highly pervasive, as suggested by high rates of encounter between males and females \citep{Kamath2018}, making the divergence of habitat-populations by drift in relative genetic isolation an unlikely scenario.\\

A number of alternative explanations remain. First, there could sexual selection for different dewlap colors in different locations. Indeed, although the sensory-drive hypothesis may include a sexual selection aspect, e.g. if the optimal male phenotype in a given habitat is a function of female perception, sexual preferences may also be arbitrary and independent of the habitat \citep{Andersson1994}, and so could differ across habitats and islands. However, one previous study has found no link between dewlap coloration and body size dimorphism in \textit{A. sagrei} (a proxy for the intensity of sexual selection) in an among-island comparison \citep{Baeckens2018}, and although within-island data are lacking, this scenario may have little plausibility.\\

Alternatively, selective pressures may be different in similar habitats from one island to another, because of other environmental variables not accounted for by the habitat-type classification we used \citep{Howard1950, Schoener1968}. The islands we sampled indeed exhibit variation in some climatic variables but also in densities of predators and anole congeners, which have all been shown to correlate with variation in \textit{A. sagrei} mean dewlap coloration among islands \citep{Vanhooydonck2009, Baeckens2018}.\\

Finally, different island-populations could also respond differently to similar selective pressures, resulting in various between-habitat divergence patterns across islands. Several factors could account for this. For example, the founder populations of each island, which we know colonized the islands independently (van de Schoot, unpublished thesis; \citealt{Driessens2017, Reynolds2020}), could have exhibited different dewlap colors at the time of colonization, as may be suggested by the larger differences we observed between than within islands. In turn, different initial phenotypes could have led to different ways in which populations would have diverged between habitats. Moreover, the different founding populations may have also consisted in different subsets of the standing genetic variation of their Cuban ancestor due to potential bottlenecks \citep{Reynolds2020}, which may have constrained the way they would later respond to the local selective pressures of their respective islands.\\

Altogether, our results show that dewlap color of \textit{A. sagrei} commonly varies between habitat-types, even in close geographical proximity, within islands of the West Indies, and that coloration differs in different ways from one island to another. We discussed several non-mutually exclusive mechanisms that could explain these observations, but more data are needed to thoroughly test each of these. Nevertheless, heterogeneous patterns of divergence across islands do not support an adaptive sensory-drive scenario, and our we propose that within-island dewlap color variation may be underlain by a more subtle mosaic of factors.

% What is missing?
% It reads quite complete, although I haven't talked about all the examples where dewlaps were found to vary between localities in the literature --- at least that would be a dilute a bit the large number of speculations in the intro.
% Also it would be nice to echo the concept of adaptation at small spatial scales talked about in the intro
% Maybe let it rest and read it all first to see what is missing
% How did closely related island-populations diverge between habitats relative to each other? e.g. Cayman Islands?