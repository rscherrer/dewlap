\begin{table}
	\label{suptab:counts}	
	\caption{Number of lizards sampled in each habitat on each island.}
	\centering
	% Sample sizes
\begin{table}[H]
    \caption{Numbers of lizards sampled across islands and habitats.}
    \centering
    \begin{tabular}{lrrr}
        \hline
        & coastal & coppice & mangrove\\
        \hline
        Abaco & 41 & 24 & 21\\
        Bimini & 38 & 14 & 15\\
        Cayman Brac & 15 & 18 & 17\\
        Eleuthera & 22 & 25 & 9\\
        Little Cayman & 17 & 12 & 16\\
        Long Island & 26 & 14 & 13\\
        North Andros & 9 & 9 & 10\\
        Ragged Island & 18 & 15 & 17\\
        South Andros & 10 & 9 & 12\\
        \hline
    \end{tabular}
    \label{suptab:counts}
\end{table}
\end{table}

\begin{table}
	\label{suptab:sites}
	\caption{Locations of the sampling sites across islands, with mean principal component scores per site.}
	\centering
	
\begin{tabular}{lrrlrrrr}
\toprule
Island & Longitude & Latitude & Habitat & PC1 & PC2 & PC3 & PC4\\
\midrule
Abaco & -77.7256 & 26.9083 & mangrove & -5.4905 & 1.3541 & -0.4741 & 0.0083\\
Abaco & -77.5800 & 26.9020 & coastal & 1.8633 & 0.0365 & -0.4475 & 0.0033\\
Abaco & -77.5763 & 26.9128 & coppice & -1.6738 & -1.7793 & -0.0499 & 0.0012\\
Abaco & -77.1784 & 26.1045 & coastal & 1.1863 & 2.0408 & -0.3468 & 0.0022\\
Abaco & -77.0055 & 26.3254 & mangrove & -9.0319 & -2.7460 & 0.4687 & 0.0077\\
Abaco & -77.0039 & 26.3170 & coppice & 0.9967 & 0.5161 & -0.0267 & -0.0118\\
Abaco & -76.9968 & 26.3260 & coastal & 7.6077 & 0.3186 & 0.1771 & -0.0008\\
Bimini & -79.3022 & 25.5859 & coastal & 5.7537 & -0.1593 & -0.2505 & 0.0001\\
Bimini & -79.3014 & 25.7052 & coastal & -3.1822 & 1.6617 & -0.0460 & 0.0024\\
Bimini & -79.3002 & 25.7042 & coppice & -1.3514 & -3.8786 & 0.1027 & -0.0027\\
Bimini & -79.2709 & 25.7066 & mangrove & 3.3656 & 0.6244 & 0.1569 & -0.0021\\
Cayman Brac & -79.8627 & 19.6878 & coastal & 6.6606 & -2.5670 & 0.0166 & -0.0007\\
Cayman Brac & -79.8441 & 19.6949 & mangrove & -1.0914 & 4.3607 & 0.0855 & 0.0001\\
Cayman Brac & -79.7887 & 19.7209 & coppice & -4.5197 & -1.9793 & -0.0946 & 0.0004\\
Eleuthera & -76.3347 & 24.8146 & coppice & 3.2669 & -1.2404 & 0.1018 & -0.0085\\
Eleuthera & -76.3058 & 24.8127 & coastal & 0.4216 & -3.5133 & -0.0567 & 0.0009\\
Eleuthera & -76.2901 & 24.7981 & mangrove & 2.1881 & 0.7517 & 0.3957 & -0.0055\\
Eleuthera & -76.1616 & 24.9129 & coppice & -1.9136 & 1.0868 & -0.4978 & -0.0092\\
Eleuthera & -76.1492 & 24.9335 & coastal & -3.1863 & 2.4270 & 0.1881 & 0.0218\\
Little Cayman & -80.0660 & 19.6906 & coppice & 0.8021 & -1.9569 & -0.0760 & -0.0068\\
Little Cayman & -80.0205 & 19.6865 & coastal & -6.6917 & -1.2615 & 0.0659 & 0.0057\\
Little Cayman & -79.9871 & 19.6986 & mangrove & 6.5083 & 2.8079 & -0.0129 & -0.0010\\
Long Island & -75.2299 & 23.4740 & mangrove & -1.2873 & 1.9371 & -0.1880 & -0.0029\\
Long Island & -75.2063 & 23.4282 & coastal & 2.3686 & -0.9033 & 0.0215 & 0.0096\\
Long Island & -75.1884 & 23.4292 & coppice & -4.6266 & 0.5060 & 0.1049 & -0.0070\\
Long Island & -75.1408 & 23.3883 & coastal & 3.6139 & -1.4521 & 0.0475 & 0.0025\\
North Andros & -77.8908 & 24.8391 & coastal & -2.1881 & -1.1236 & 0.0397 & -0.0060\\
North Andros & -77.8428 & 24.7516 & coppice & -1.8115 & 0.0012 & -0.1678 & 0.0024\\
North Andros & -77.7540 & 24.6644 & mangrove & 3.5997 & 1.0101 & 0.1153 & 0.0033\\
Ragged Island & -75.7364 & 22.1768 & coppice & 3.2851 & -0.3274 & 0.1911 & -0.0013\\
Ragged Island & -75.7314 & 22.2097 & coastal & -0.6412 & -0.8878 & -0.1293 & -0.0033\\
Ragged Island & -75.7276 & 22.2045 & mangrove & -2.9188 & 1.5792 & -0.0034 & 0.0099\\
Ragged Island & -75.7270 & 22.1973 & mangrove & -1.2210 & 0.7285 & -0.0721 & -0.0028\\
South Andros & -77.6050 & 24.2027 & mangrove & -3.9253 & 0.4734 & 0.0477 & -0.0005\\
South Andros & -77.5936 & 24.1289 & coppice & 6.1152 & -0.4925 & 0.0349 & 0.0012\\
South Andros & -77.5453 & 24.0764 & coastal & -0.7933 & -0.1248 & -0.0887 & -0.0004\\
\bottomrule
\end{tabular}

\end{table}

\begin{table}
	\label{suptab:pcavariances}
	\caption{Proportion of variance explained by the first four principal components on each island, as well as across the whole archipelago.}
	\centering
	
\begin{tabular}{l|r|r|r|r|r}
\hline
island & PC1 & PC2 & PC3 & PC4 & total\\
\hline
Abaco & 0.400 & 0.279 & 0.147 & 0.079 & 0.906\\
\hline
Bimini & 0.502 & 0.208 & 0.160 & 0.051 & 0.921\\
\hline
Cayman Brac & 0.438 & 0.190 & 0.155 & 0.105 & 0.888\\
\hline
Eleuthera & 0.490 & 0.233 & 0.138 & 0.066 & 0.926\\
\hline
Little Cayman & 0.441 & 0.212 & 0.176 & 0.078 & 0.907\\
\hline
Long Island & 0.515 & 0.205 & 0.161 & 0.043 & 0.925\\
\hline
North Andros & 0.560 & 0.170 & 0.152 & 0.054 & 0.937\\
\hline
Ragged Island & 0.483 & 0.226 & 0.127 & 0.072 & 0.907\\
\hline
South Andros & 0.488 & 0.247 & 0.146 & 0.067 & 0.948\\
\hline
Archipelago & 0.473 & 0.197 & 0.164 & 0.079 & 0.913\\
\hline
\end{tabular}

\end{table}

\begin{table}
	\label{suptab:brightness}
	\caption{Pearson's correlation test between dewlap brightness, as measured by the average reflectance between 300 and 700nm in wavelength, and PC1 scores, for all islands and across the whole archipelago. ***, P < 0.001.}
	\centering
	
\begin{tabular}{l|r|l|l}
\hline
island & r2 & pvalue & \\
\hline
Abaco & 0.908 & < 0.0001 & ***\\
\hline
Bimini & 0.999 & < 0.0001 & ***\\
\hline
Cayman Brac & 0.987 & < 0.0001 & ***\\
\hline
Eleuthera & 0.963 & < 0.0001 & ***\\
\hline
Little Cayman & 0.965 & < 0.0001 & ***\\
\hline
Long Island & 0.986 & < 0.0001 & ***\\
\hline
North Andros & 0.994 & < 0.0001 & ***\\
\hline
Ragged Island & 0.978 & < 0.0001 & ***\\
\hline
South Andros & 0.979 & < 0.0001 & ***\\
\hline
Archipelago & 0.976 & < 0.0001 & ***\\
\hline
\end{tabular}

\end{table}

\begin{table}
	\label{suptab:multinorm}
	\caption{Henze-Zirkler's test of multivariate normality, performed on principal components in each habitat and on each island. HZ, test statistic. *, P < 0.05; **, P < 0.01; ***, P < 0.001.}
	\centering
	
\begin{tabular}{llrrl}
\toprule
island & habitat & HZ & pvalue & signif\\
\midrule
Abaco & coastal & 1.10 & 0.0027 & **\\
Abaco & coppice & 1.07 & 0.0022 & **\\
Abaco & mangrove & 1.06 & 0.0023 & **\\
Bimini & coastal & 1.28 & 0.0001 & ***\\
Bimini & coppice & 0.85 & 0.0482 & *\\
\addlinespace
Bimini & mangrove & 1.19 & 0.0001 & ***\\
Cayman Brac & coastal & 0.65 & 0.5311 & \\
Cayman Brac & coppice & 0.70 & 0.3940 & \\
Cayman Brac & mangrove & 0.66 & 0.5357 & \\
Eleuthera & coastal & 1.61 & 0.0000 & ***\\
\addlinespace
Eleuthera & coppice & 1.48 & 0.0000 & ***\\
Eleuthera & mangrove & 0.73 & 0.1423 & \\
Little Cayman & coastal & 0.62 & 0.6599 & \\
Little Cayman & coppice & 0.64 & 0.4867 & \\
Little Cayman & mangrove & 0.87 & 0.0413 & *\\
\addlinespace
Long Island & coastal & 0.82 & 0.1468 & \\
Long Island & coppice & 0.92 & 0.0150 & *\\
Long Island & mangrove & 0.77 & 0.1289 & \\
North Andros & coastal & 0.66 & 0.3174 & \\
North Andros & coppice & 0.76 & 0.0900 & \\
\addlinespace
North Andros & mangrove & 0.67 & 0.3185 & \\
Ragged Island & coastal & 0.76 & 0.2268 & \\
Ragged Island & coppice & 0.80 & 0.1115 & \\
Ragged Island & mangrove & 0.54 & 0.9022 & \\
South Andros & coastal & 0.66 & 0.3451 & \\
\addlinespace
South Andros & coppice & 0.66 & 0.3154 & \\
South Andros & mangrove & 0.91 & 0.0144 & *\\
\bottomrule
\end{tabular}

\end{table}

\begin{table}
	\label{suptab:covariance}
	\caption{Box's M-test of homogeneity of covariance matrices across habitats on each island. $\chi^2$, test statistic. *, P < 0.05; **, P < 0.01; ***, P < 0.001.}
	\centering
	
\begin{tabular}{lrrrl}
\toprule
island & chisq & df & pvalue & signif\\
\midrule
Abaco & 47.1 & 20 & 0.0006 & ***\\
Bimini & 36.0 & 20 & 0.0152 & *\\
Cayman Brac & 36.9 & 20 & 0.0120 & *\\
Eleuthera & 44.6 & 20 & 0.0013 & **\\
Little Cayman & 32.8 & 20 & 0.0356 & *\\
\addlinespace
Long Island & 56.2 & 20 & 0.0000 & ***\\
North Andros & 33.7 & 20 & 0.0283 & *\\
Ragged Island & 29.3 & 20 & 0.0824 & \\
South Andros & 46.5 & 20 & 0.0007 & ***\\
\bottomrule
\end{tabular}

\end{table}

\begin{table}
	\label{suptab:normality}
	\caption{Shapiro-Wilk's test of univariate normality performed on each island where significant differences were detected by SVM classification, in each habitat where deviations from multivariate normality were detected. $W$, test statistic. *, P < 0.05; **, P < 0.01; ***, P < 0.001.}
	\centering
	
\begin{tabular}{lllrrl}
\toprule
island & habitat & variable & W & pvalue & signif\\
\midrule
Abaco & coastal & PC1 & 0.954 & 0.0941 & \\
Abaco & coastal & PC2 & 0.927 & 0.0112 & *\\
Abaco & coastal & PC3 & 0.973 & 0.4228 & \\
Abaco & coastal & PC4 & 0.955 & 0.1027 & \\
Abaco & coppice & PC1 & 0.970 & 0.6776 & \\
\addlinespace
Abaco & coppice & PC2 & 0.816 & 0.0005 & ***\\
Abaco & coppice & PC3 & 0.930 & 0.0976 & \\
Abaco & coppice & PC4 & 0.941 & 0.1711 & \\
Abaco & mangrove & PC1 & 0.881 & 0.0155 & *\\
Abaco & mangrove & PC2 & 0.869 & 0.0093 & **\\
\addlinespace
Abaco & mangrove & PC3 & 0.986 & 0.9873 & \\
Abaco & mangrove & PC4 & 0.939 & 0.2044 & \\
Bimini & coastal & PC1 & 0.821 & 0.0000 & ***\\
Bimini & coastal & PC2 & 0.960 & 0.1854 & \\
Bimini & coastal & PC3 & 0.856 & 0.0002 & ***\\
\addlinespace
Bimini & coastal & PC4 & 0.945 & 0.0611 & \\
Bimini & coppice & PC1 & 0.911 & 0.1648 & \\
Bimini & coppice & PC2 & 0.958 & 0.6927 & \\
Bimini & coppice & PC3 & 0.953 & 0.6146 & \\
Bimini & coppice & PC4 & 0.971 & 0.8953 & \\
\addlinespace
Bimini & mangrove & PC1 & 0.884 & 0.0536 & \\
Bimini & mangrove & PC2 & 0.976 & 0.9363 & \\
Bimini & mangrove & PC3 & 0.982 & 0.9805 & \\
Bimini & mangrove & PC4 & 0.975 & 0.9232 & \\
Eleuthera & coastal & PC1 & 0.909 & 0.0461 & *\\
\addlinespace
Eleuthera & coastal & PC2 & 0.886 & 0.0157 & *\\
Eleuthera & coastal & PC3 & 0.906 & 0.0390 & *\\
Eleuthera & coastal & PC4 & 0.962 & 0.5293 & \\
Eleuthera & coppice & PC1 & 0.922 & 0.0567 & \\
Eleuthera & coppice & PC2 & 0.954 & 0.3055 & \\
\addlinespace
Eleuthera & coppice & PC3 & 0.781 & 0.0001 & ***\\
Eleuthera & coppice & PC4 & 0.901 & 0.0188 & *\\
Little Cayman & mangrove & PC1 & 0.907 & 0.1024 & \\
Little Cayman & mangrove & PC2 & 0.904 & 0.0924 & \\
Little Cayman & mangrove & PC3 & 0.739 & 0.0005 & ***\\
\addlinespace
Little Cayman & mangrove & PC4 & 0.973 & 0.8802 & \\
Long Island & coppice & PC1 & 0.686 & 0.0003 & ***\\
Long Island & coppice & PC2 & 0.848 & 0.0210 & *\\
Long Island & coppice & PC3 & 0.931 & 0.3188 & \\
Long Island & coppice & PC4 & 0.904 & 0.1280 & \\
\addlinespace
South Andros & mangrove & PC1 & 0.787 & 0.0067 & **\\
South Andros & mangrove & PC2 & 0.861 & 0.0500 & *\\
South Andros & mangrove & PC3 & 0.697 & 0.0008 & ***\\
South Andros & mangrove & PC4 & 0.950 & 0.6411 & \\
\bottomrule
\end{tabular}

\end{table}

\begin{table}
	\label{suptab:anova-pooled}
	\caption{Univariate ANOVAs performed on each principal component across the whole archipelago. Legend is the same as for Table \ref{tab:anova}, except that best fitting models 3 and 4 refer to the mixed effect equivalents to the OLS and GLS model, with island as a random effect (see Methods).}
	\centering
	
\begin{tabular}{lrrrrrrrrrl}
\toprule
variable & best\_fit & df\_model & AICc & dAICc & AICcw & df\_LRT & loglik & lratio & pvalue & signif\\
\midrule
PC1 & 3 & 5 & 3749.9 & -228.3 & 0.613 & 2 & -1874.7 & 8.69 & 0.0130 & *\\
PC2 & 4 & 7 & 3002.2 & -162.3 & 0.976 & 2 & -1496.2 & 17.76 & 0.0001 & ***\\
PC3 & 4 & 7 & 2826.3 & -175.4 & 0.968 & 2 & -1407.8 & 7.03 & 0.0298 & *\\
PC4 & 4 & 7 & 2015.7 & -305.8 & 0.519 & 2 & -1000.1 & 0.47 & 0.7914 & \\
\bottomrule
\end{tabular}

\end{table}

\begin{table}
	\label{suptab:classif-svm-pca}
	\caption{Mean SVM classification accuracy per island, over all replicates and cross-validation bins. $N$, number of observations per island; $p_{\mbox{test}}$, proportion of the data sampled to form the training set; $n_{\mbox{test}}$, number of observations in the testing set. P-values indicate deviations from the expected null binomial distribution, with $n_{\mbox{test}}$ events per island and random guess success probability $1/3$. *, P < 0.05, **, P < 0.01, ***, P < 0.001.}
	\centering
	
\begin{tabular}{lrrrrrl}
\toprule
Island & Accuracy & N & $p_{\mbox{test}}$ & $n_{\mbox{test}}$ & $P$ & \\
\midrule
Abaco & 0.612 & 86 & 0.2 & 17 & 0.0080 & **\\
Bimini & 0.547 & 67 & 0.2 & 13 & 0.0347 & *\\
Cayman Brac & 0.721 & 50 & 0.2 & 10 & 0.0034 & **\\
Eleuthera & 0.437 & 56 & 0.2 & 11 & 0.2890 & \\
Little Cayman & 0.734 & 45 & 0.2 & 9 & 0.0083 & **\\
Long Island & 0.651 & 53 & 0.2 & 10 & 0.0197 & *\\
North Andros & 0.453 & 28 & 0.2 & 5 & 0.2099 & \\
Ragged Island & 0.364 & 50 & 0.2 & 10 & 0.4407 & \\
South Andros & 0.600 & 31 & 0.2 & 6 & 0.1001 & \\
\bottomrule
\end{tabular}

\end{table}

\begin{table}
	\label{suptab:kruskal}
	\caption{Results of nonparametric Kruskal-Wallis tests performed on each variable on each island where deviations from normality were detected.}
	\centering
	
\begin{tabular}{llrrrl}
\toprule
Island & Variable & $\chi^2$ & df & $P$ & \\
\midrule
Abaco & PC1 & 0.74 & 2 & 0.6924 & \\
Abaco & PC2 & 23.13 & 2 & 0.0000 & ***\\
Bimini & PC1 & 7.38 & 2 & 0.0250 & *\\
Bimini & PC3 & 15.17 & 2 & 0.0005 & ***\\
Little Cayman & PC3 & 19.95 & 2 & 0.0000 & ***\\
Long Island & PC1 & 10.98 & 2 & 0.0041 & **\\
Long Island & PC2 & 4.02 & 2 & 0.1339 & \\
\bottomrule
\end{tabular}

\end{table}

\begin{table}
	\label{suptab:autocor}
	\caption{Individual-based permutation tests of spatial autocorrelation within islands. P-values were computed from 1,000 permutations of individual site-labels. Pearson's coefficient $r$ measures the correlation between distances in color space and geodesic distances among the sites. $N$, number of sites. *, P < 0.05.}
	\centering
	
\begin{tabular}{lrrrl}
\toprule
Island & $r^2$ & $P$ & N & \\
\midrule
Abaco & -0.213 & 0.817 & 7 & \\
Bimini & 0.044 & 0.510 & 4 & \\
Cayman Brac & -0.010 & 0.465 & 3 & \\
Eleuthera & 0.816 & 0.015 & 5 & *\\
Little Cayman & -0.688 & 0.684 & 3 & \\
Long Island & -0.189 & 0.579 & 4 & \\
North Andros & 0.730 & 0.199 & 3 & \\
Ragged Island & 0.706 & 0.114 & 4 & \\
South Andros & -0.852 & 0.776 & 3 & \\
\bottomrule
\end{tabular}

\end{table}