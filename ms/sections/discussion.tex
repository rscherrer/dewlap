
\paragraph{Dewlap coloration differs between habitats, but the differentiation is inconsistent with adaptive convergence} Dewlaps are highly conspicuous in anoles and used as communication signals in courtship \citep{Sigmund1983, Driessens2014, Driessens2015} and territorial displays \citep{Losos1985, Macedonia1994, Macedonia2013}. On five of the nine islands we sampled in the West Indies, we found that \textit{A. sagrei} male dewlap coloration significantly varied between individuals living in three different habitat types  (beach scrub bush, primary coppice forest and mangrove forest), even though those habitats were in close proximity to each other and the \textit{sagrei} populations are continuously distributed between the habitats.\\ 

One compelling indicator of adaptation by natural selection is the repetition of patterns of differentiation across localities \citep{Losos2009, Losos2011}. In the context of this study, had we found consistent patterns of dewlap differentiation between habitats, we would have concluded that there was likely an adaptive basis for this pattern. However, the inconsistent and idiosyncratic patterns of dewlap variation we observed contradict this prediction. What then explains the observed differences among the populations?\\

One explanation is that this pattern is the result of multiple island colonizations of phenotypically different individuals to these habitats. However, we reject this possibility because all of the island populations in this study are strictly monophyletic, reflecting a single colonization event per island (van de Schoot, unpublished thesis; \citealt{Driessens2017, Reynolds2020}). Other drivers could be at work, including genetic drift, phenotypic plasticity, and divergent natural selection. Of these, we find divergent natural selection the most likely explanation, suggesting that future research with additional granularity on the drivers of dewlap coloration is needed. 

\paragraph{A role of neutral drift is unlikely} Genetic drift may lead to patterns of trait divergence similar to local adaptation \citep{Miles2019}. We think this scenario is implausible, however, as we found significant differences in dewlap color over distances too small for genetic drift to have realistically counteracted the effect of homogenizing gene flow \citep{Richardson2014}. On Bimini, for example, we detected differences between sites only a few hundred meters away and \textit{A. sagrei}, have previously been shown to be highly mobile \citep{Kamath2018}. Moreover, habitat-populations within each island were found to be non-monophyletic and often share identical haplotypes, based on phylogenetic analysis of mitochondrial DNA (van de Schoot et al. unpublished thesis). This is further evidence of gene flow between habitats.\\

It is possible that complex drift scenarios other than simple distance-mediated, diffusion-like processes may be at play \citep{Miles2019}. For example,  we did detect a significant signal of isolation-by-distance on Eleuthera, without finding differences in dewlap coloration between habitats, suggesting a possible role of drift on this island. Nonetheless, by and large our results align with previously documented cases of highly localized  dewlap color divergence despite gene flow in multiple species of anoles. \citet{Ng2012} and \citet{Ng2016} found divergent dewlap coloration despite gene flow between subspecies of \textit{A. distichus} on Hispaniola, and proposed this as a mechanism of reproductive isolation in the early stages of speciation \citep{Ng2011, Lambert2013, Ng2017}. \citet{Stapley2011} found that dewlap color polymorphism was maintained in the absence of genetic structure between populations of \textit{A. apletophallus} from Panama. \citet{Thorpe2002a} found that divergence in dewlap coloration matched habitat-type better than mitochondrial lineage in \textit{A. roquet} on Martinique, and a convergent pattern was found in \textit{A. trinitatis} on the featureless island of St Vincent \citep{Thorpe2002b}. Finally, regionally-distinct body coloration, but not dewlap coloration, is present in \textit{A. conspersus} on another small island, Grand Cayman, where no physical barriers to gene flow exist \citep{Macedonia2001}.

\paragraph{A role of phenotypic plasticity is unlikely} Differences in coloration between habitat populations may not be genetically determined and instead may be influenced by environmental factors. The yellow, orange and red coloration in anoline dewlaps are produced by pterins and carotenoids \citep{Ortiz1962, Ortiz1962a, Ortiz1963, Ortiz1966, Macedonia2000, Steffen2007, Steffen2009}. Animals lack the ability to synthesize carotenoids, and those must therefore be found in the diet, while pterins are synthesized from nucleotides \citep{Goodwin1984, Hill2002, Hill2006}. However, experimental manipulation of dietary carotenoid content showed no effect on dewlap coloration in \textit{A. sagrei} \citep{Steffen2010} nor in \textit{A. distichus} \citep{Ng2013}, which also has an orange-based dewlap. (Dewlap color has also been shown to change in relation to parasite load \citep{Cook2013}, though parasites were not investigated in this study. -- to remove) These laboratory results  suggest a plastic response in dewlap color to differences in diet across habitats unlikely, suggesting the differences we observed could therefore be genetically based. This hypothesis is further supported by \citet{Cox2017}, who found a high degree of heritability of dewlap coloration in \textit{A. sagrei}.

\paragraph{Sensory Drive - maybe drop?} One hypothesis that a relationship would exist between dewlap color and habitat stems from the idea that the communication signals evolve adaptively in response to light environment (the sensory drive hypothesis, \citealt{Endler1988, Endler1992, Endler1998}). However, we find this explanation unlikely because the differences we observed were both inconsistent among islands and inconsistent with predictions of the sensory drive hypothesis. Previous studies have proposed that dewlap coloration may have evolved to be maximally detectable under local light conditions, primarily through UV contrast (i.e. UV-brighter dewlaps in UV-dark, mesic habitats and UV-darker dewlaps in UV-bright, xeric habitats), in \textit{ A. cristatellus} and \textit{A. cooki} from Puerto Rico \citep{Leal2002, Leal2004}. On the contrary, we found no apparent habitat-dependent maximization of UV-contrast, or just any contrast in \textit{ A. sagrei}. Instead, we found for example the darkest dewlaps in the dark, mesic habitat -- primary coppice forest -- on three islands, and dewlaps often differed the most between beach scrub and mangrove forest, two xeric habitats with similar, high irradiance levels \citep{Howard1950, Schoener1968}. Studies of Jamaican and Hispaniolan anoles similarly found between-habitat differences in dewlap coloration but no evidence for higher dewlap detectability in different habitats \citep{Fleishman2009, Ng2012}. Our data are consistent with those previous results in suggesting that adaptation to local light conditions, or at least broad habitat types, is not a major driver of the within-island variation in dewlap coloration in \textit{A. sagrei}.

\paragraph{Fisherian sexual selection - maybe drop?} Selection, however, needs not necessarily be linked to habitat type, and may take the form of arbitrary, "Fisherian" sexual selection, where female preferences differ between localities for reasons unrelated to the environment, driving the divergent evolution of male ornaments \citep{Lande1981, Andersson1994, Higashi1999}. This process could account for the idiosyncratic patterns of within-island divergence we report, where initial differences in female preferences could have arisen for nonselective reasons (e.g. plasticity or random drift). Substantial levels of promiscuity in \textit{A. sagrei} suggest ample opportunity for female mate choice \citep{Kamath2018}, and are in line with this scenario. However, \citep{Baeckens2018} found no link between \textit{A. sagrei} dewlap coloration and size dimorphism (a proxy for sexual selection) in an among-island study of the same archipelagos.

\paragraph{Natural selection may proceed differently across islands and habitats} A further potential explanation for the inconsistencies in dewlap coloration among habitats between islands is that the observed differences in dewlap color are the result of divergent natural selection. If that is the case, however, either the habitats we have identified are not as similar as we thought, such that the same habitat type on different islands favors different colors, or, alternatively, the selective pressures causing dewlap divergence are related to some environmental factor other than habitat type that we did not measure. The habitats on different islands may, for example, differ in densities of predators or congeners, which have been shown to affect among-island dewlap diversity \cite{Vanhooydonck2009, Baeckens2018}. In particular, \citet{Baeckens2018} recently showed that dewlaps with spotted patterns occurred more often in \textit{A. sagrei} on islands with more coexisting species of anoles. Therefore, if local adaptation occurs, it may be more likely to involve components of the environment that do not encompass our broad habitat categories leading to island-specific differences between habitats rather than parallel differences expected if habitat identity was the primary selective pressure driving divergence.\\ 

One last possibility to explain idiosyncratic patterns of between-habitat differences we observe is that, if island populations are different in color on average, they may be experiencing different portions of a within-island fitness landscape whose shape changes depending on the starting phenotype. In this study, we indeed found that dewlap coloration differed much more between than within islands. So, a population may not respond to selection in the same manner depending on its starting conditions. This is true not only in terms of phenotype (average dewlap color differs among islands) but also genetic composition (e.g. independent colonizations may have important genetic founder effects, \citealt{Reynolds2020}), and so the selective pressures applying to populations with different starting phenotypes may drive these divergent outcomes. For example, \citet{Fleishman2020} found that yellow stimuli are less detectable than red stimuli in high-light environments while both colors are readily detectable in low-light environments. Consequently, the environmental pressures applying to yellower dewlaps may differ than those for redder dewlaps, as red dewlaps should be always equally or more detectable than yellow dewlaps across light environments, whereas yellow dewlaps could experience more directional selection towards more red when in a high-light environments.

\paragraph{Conclusion} We identified patterns of dewlap color differentiation in populations of \textit{A. sagrei} from different habitat-types on multiple small islands of the West Indies. However, our main hypothesis to explain our findings -- local adaptation and sensory drive -- received little support from our multiple-island dataset. We also found other mechanisms such as drift or plasticity unlikely candidates to explain the observed patterns. This suggests that combinations of multiple factors interacting in subtle ways, and not necessarily in parallel across islands, may have contributed to present-day dewlap color diversity. 
