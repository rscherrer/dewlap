Two main insights follow from our results. First, we detected significant differences in dewlap coloration between habitats within seven out of the nine islands investigated (excluding North Andros where the follow-up univariate analyses were not significant), suggesting a putatively high potential for local differentiation of dewlap coloration in \textit{Anolis sagrei}. Second, we found differences in coloration along different dimensions of color space, and in different directions, on different islands.\\

Detectable differences in dewlap color between populations are surprising, as habitats were often in close geographical proximity to each other (as close as a few hundred meters on Bimini and most of the time within ten kilometers)\textcolor{red}{. Indeed, given that (1) the populations were continuously distributed between the habitats, (2) populations from different habitats were not monophyletic with respect to mitochondrial haplotypes (\citealt{vandeSchoot2016} unpublished thesis), and (3) \textit{A. sagrei} is highly mobile \citep{Kamath2018}, we would have expected more homogeneous distributions of color phenotypes within islands due to gene flow, with fewer differences between populations, especially those in close proximity.} \textcolor{olive}{, and we would have expected gene flow to cause a more homogeneous distriution of color phenotypes within islands. While little is known about the cruising range of individuals from our study populations (but see \citealt{Steinberg2017, Kamath2018}), \textit{A. sagrei} are polygynandrous (where both males and females mate with multiple mates, \citealt{Kamath2017, Kamath2018, Kamath2018a}), thus offering opportunity for gene flow, especially given that lizards were distributed continuously and at high densities within the islands we sampled. Consistent with that, while populations from different islands were monophyletic, individuals within islands were not monophyletic with respect to habitat based on mitochondrial haplotypes (\citealt{vandeSchoot2016} unpublished thesis).}\\

Several scenarios could account for these findings. One explanation is an adaptive one: populations living in different habitats could be phenotypically adapted to their local environmental conditions. Given that the brightly colored dewlap of \textit{A. sagrei} is used as a communication signal, its color may be a target for selection if the transmission or perception of the signal differs from one habitat to another, for example because of differences in ambient light, according to the sensory-drive hypothesis \citep{Endler1988, Endler1992, Endler1998}. The sensory-drive hypothesis has been tested multiple times for dewlap coloration in \textit{Anolis} lizards, with mixed results. Some authors found support for it \citep{Leal2002, Leal2004}, while others found differences in dewlap coloration between habitats inconsistent with the sensory-drive hypothesis \citep{Fleishman2009, Ng2012}.\\

If our results were an example of sensory drive, we would have expected to see consistent differences between populations from different habitats across islands, given the apparent environmental consistency each of the three habitat types across the islands we sampled. In particular, we would have expected divergence in line with increased detectability given local light conditions, such as the high contrasts with background vegetation found in the UV range in \citet{Leal2002} and \citet{Leal2004}. We might also have expected mangrove and beach scrub lizards, both inhabiting areas with high light penetration, to have more similar dewlaps, and to differ significantly from lizards from the coppice habitat, where irradiance is low. Instead, we found inconsistencies in the way dewlap color differed between habitats across islands. While short-wavelengths (UV reflectance) were often involved in color differences, they were not involved on all islands where significant differences were detected. On some islands, other or additional variables differed, such as brightness, red reflectance or the reflectance at the ends of the spectrum visible to \textit{Anolis} lizards (UV and red, \citealt{Lazareva2012}) relative to intermediate wavelengths (blue-to-yellow). Similar portions of the spectrum were sometimes involved in opposite directions on different islands, such as on Abaco and Cayman Brac, where mangrove lizards had a higher UV-reflectance than beach scrub lizards on the former, but a lower UV-reflectance on the latter. Overall, the observed heterogeneity of divergence patterns across islands provides no support to a sensory-drive explanation.\\

It is presently not known if the reported differences in coloration have a genetic basis. Yet, we find it unlikely that these differences arose through phenotypic plasticity, as although the carotenoids that partly make up the red and orange colors of anole dewlaps must be found in the diet \citep{Goodwin1984, Hill2002, Hill2006}, studies testing the effect of carotenoid deprivation \citep{Steffen2010, Ng2013} and heritability \citep{Cox2017} of dewlap coloration in \textit{A. sagrei} and \textit{A. distichus} (another species with a carotenoid-based dewlap), found little support for phenotypic and developmental plasticity in dewlap coloration. One exception is a study demonstrating that lizards heavily parasitized by skin mites had duller dewlaps \citep{Cook2013}, but we found no sign of that in our study.\\

\sout{We found no evidence for a role of genetic drift in explaining the observed patterns either. First, \textit{A. sagrei} was distributed across the islands continuously, usually at relatively high population densities, rather than in small and isolated populations where drift might be expected to have a strong effect. Second, we found no evidence of isolation-by-distance except on Abaco, which was sampled at the largest geographical scale, with sites nearly a hundred kilometers apart from each other. Hence, while isolation-by-distance may explain long-range differences on this island, most of the differences among habitats across the rest of the sampling region are unlikely to be explained by genetic drift, as habitats were often in close proximity (less than 10km).}\\

\textcolor{olive}{Genetic drift could contribute to some of the observed variation. Indeed, while only Abaco showed significant patterns consistent with isolation-by-distance (which may emerge under limited dispersal and drift, \citealt{Wright1943, Kimura1964, Slatkin1987}), there may have been too few sites on most islands to conclusively detect it, and even then, the absence of detectable isolation-by-distance may not necessarily constitute evidence for the absence of drift. Besides, spatial autocorrelation was the strongest on islands sampled at the largest scales (e.g. some sites on Abaco were nearly 100km apart, and Eleuthera -- the second strongest signal, albeit nonsignificant -- had sites more than 30km apart), such that it is possible that neutral processes and/or dispersal limitations might contribute to shaping variation over long distances. That said, many significant differences were found between habitats in close proximity, contrary to what would be expected under isolation-by-distance, including on islands where spatial autocorrelation was detected. Moreover, \textit{A. sagrei} was distributed across the islands continuously, usually at relatively high population densities, rather than in small and isolated populations particularly prone to drift. Together with the fact that isolation-by-distance may not necessarily only emerge from drift (e.g. if there is a spatial environmental gradient), this indicates that genetic drift may have limited potential to explain the differences observed between habitats, at least at a local scale.}\\

In this study, we found larger differences among than within islands, a pattern already reported and linked to climatic conditions \citep{Driessens2017} and to densities of predators and of anole congeners \citep{Vanhooydonck2009, Baeckens2018}. Differences among habitats within islands, however, are still difficult to account for. Remaining hypotheses may include, for example, runaway sexual selection \textcolor{olive}{(i.e. arbitrary preferences of females for some colors over others, \citealt{Andersson1994})} \sout{arbitrarily} operating in different directions across islands, but no evidence so far suggests that dewlap is a target of mate choice in anoles \citep{Tokarz2002, Tokarz2005, Lailvaux2006, Nicholson2007}. Another hypothesis is that the different genetic constitutions of different islands, perhaps resulting from founder effects (the islands have been colonized independently, \citealt{vandeSchoot2016} unpubl.; \citealt{Driessens2017, Reynolds2020}), may have predisposed populations to adapt differently to similarly selective circumstances. \textcolor{olive}{That said, signalling can be a multifacetted behavior, and while we focused on dewlap color here, other potentially important elements of behavioral interactions include dewlap size, dewlap patterning, display activity as well as behaviors associated with dewlap extensions such as headbobs and pushups \citep{Vanhooydonck2005, Driessens2014, Driessens2015, Lailvaux2015}. These have been linked in various ways (and not always consistently across studies) not only with environmental variables but also with proxies for sexual selection, predation and species recognition \citep{Vanhooydonck2005, Lailvaux2007, Vanhooydonck2009, Driessens2017, Baeckens2018}, which do vary among islands but may also vary within islands, possibly in interaction with dewlap color. Moreover, while the three habitats were clearly recognizable and consistent across islands to the human eye (Fig. \ref{fig:overview}), we did not precisely quantify irradiance within each habitat, and so we may have missed differences between islands.} Either way, new data would be needed to test these hypotheses. \textcolor{olive}{Visual modeling, for instance, would be a valuable follow-up analysis as it would indicate whether the differences we detected may be detectable by the organisms themselves (as has been done in \textit{Anolis} in \citealt{Leal2004, Fleishman2020}), but such approach requires irradiance profiles from the habitats in the field, which we presently do not have.}\\

Altogether, our results show that dewlap color of \textit{A. sagrei} commonly varies between habitat types, even in close geographical proximity, within islands of the West Indies. However, coloration differs in different ways across similar habitats from one island to another. We discussed several non-mutually exclusive mechanisms that could explain these observations. Nevertheless, heterogeneous patterns of divergence across islands do not support an adaptive sensory-drive scenario, and we propose that within-island dewlap color variation may be underlain by a more subtle mosaic of factors.