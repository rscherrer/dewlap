
% There are inconsistent differences that are not explained by distance

\paragraph{Dewlap coloration differs between habitat-types} We found that male dewlap coloration in \textit{A. sagrei} significantly varied between fine-scale habitat-types (beach scrub bush, primary coppice forest and mangrove forest) on five islands of the West Indies: Abaco, Bimini, Cayman Brac, Little Cayman and Long Island. However, the habitat-specific variation in dewlaps was not consistent between these islands. Although those results are consistent with adaptation at a very local scale, other evolutionary drivers could be at work, including phenotypic plasticity, random drift, or historical contingency, including multiple colonization events. We reject this last explanation because all of the island populations in this study are strictly monophyletic, implying a single colonization event per island (van de Schoot, unpublished thesis; \citealt{Driessens2017, Reynolds2020}).

\paragraph{A role of neutral drift is unlikely} Differences in organismal traits between environments are not necessarily proof of adaptation or selection, and genetic drift may result in patterns similar to local adaptation \citep{Miles2019}. Nevertheless, two lines of evidence from our data suggest that this scenario may be implausible. First, we found little evidence for a role of phenotypic isolation-by-distance (spatial autocorrelation) in explaining the differences we report. We did detect a significant signal of isolation-by-distance on Eleuthera, but there were no differences in dewlap coloration between habitats on this island. Second, we detected differences between habitats at relatively small spatial scales, most of the time between sites 5-10km apart, sometimes a few hundred meters away, on Bimini for example. Such small-scale differences would be unlikely under strong gene flow \citep{Richardson2014}. Our study islands lack geographic barriers to the movement of $A. sagrei$, which have been shown to be highly mobile on these islands \citep{Kamath2018}, implying widespread gene flow across sites and habitats. Moreover, habitat-populations within each island were found to be non-monophyletic admixtures of multiple haplotypes, based on phylogenetic analysis of mitochondrial DNA sequences (van de Schoot et al. unpublished thesis), suggesting widespread gene flow.\\

Our results align with previous documented cases of persistence of dewlap color divergence despite gene flow in multiple species of anoles, sometimes in relation to environmental conditions. \citet{Ng2012} and \citet{Ng2016} found divergent dewlap coloration in the face of gene flow between subspecies of \textit{A. distichus} across Hispaniola, and proposed this as a mechanism of reproductive isolation in the early stages of speciation \citep{Ng2011, Lambert2013, Ng2017}. \citet{Stapley2011} found that dewlap color polymorphism was maintained in the absence of genetic structure between populations of \textit{A. apletophallus} from Panama. \citet{Thorpe2002a} found that divergence in dewlap coloration matched habitat-type better than mitochondrial lineage in \textit{A. roquet} on Martinique, and a convergent pattern was found in \textit{A. trinitatis} on the featureless island of St Vincent \citep{Thorpe2002b}. Finally, divergence in body coloration, but not dewlap coloration, was also reported in \textit{A. conspersus} on another small island, Grand Cayman \citep{Macedonia2001}.\\

\paragraph{Dewlap coloration could be locally adapted} Although phenotypic divergence at small spatial scales may be an indicator of local adaptation, the inconsistency of the between-habitat divergence patterns we observed across islands implies that neutral processes cannot be definitely ruled out \citep{Losos2011}. Conversely, the absence of parallel divergence across islands does not rule out local adaptation either \citep{Losos2011}. If selection has an effect, it may be dependent on aspects of the environment that are not encompassed by our broad habitat-type classification into coastal dry scrub, primary coppice forest and mangrove forest, and dewlap coloration may be influenced by a mosaic of interacting local environmental factors, which need not be the same across islands.\\

Previous studies have described convergent patterns of dewlap color evolution in similar environments across islands and species \citep{Thorpe2002a, Thorpe2002b}. Others have suggested that dewlap coloration may have evolved to be maximally detectable under local light conditions imposed by the environment, primarily through UV contrast (i.e. UV-brighter dewlaps in UV-dark, mesic habitats and UV-darker dewlaps in UV-bright, xeric habitats), in \textit{A. cristatellus} and \textit{A. cooki} from Puerto Rico \citep{Leal2002, Leal2004}. Although UV reflectance was commonly involved in between-habitat divergence, we found no such patterns in \textit{A. sagrei}, where instead, we found the darkest dewlaps in the darkest, mesic habitat -- primary coppice forest -- on three islands, and dewlaps often differed the most between beach scrub and mangrove forest, two xeric habitats with similar, high irradiance levels \citep{Howard1950, Schoener1968}. The inconsistent and idiosyncratic patterns we observed suggest that dewlap color variation between habitats cannot be predicted by habitat identity alone. Studies of Jamaican and Hispaniolan anoles similarly found between-habitat differences in dewlap coloration but no evidence for higher detectability \citep{Fleishman2009, Ng2012}. Habitats on different islands may also differ in other aspects than light conditions such as densities of predators or other anole species, which have been shown to affect among-island dewlap diversity \citep{Vanhooydonck2009, Baeckens2018}. In particular, \citet{Baeckens2018} recently showed that dewlaps with spotted patterns occurred more often in \textit{A. sagrei} on islands with more coexisting species of anoles. Our data is consistent with those previous results in suggesting that adaptation to local light conditions, or at least broad habitat types, may not be a major driver of the variation in dewlap coloration in \textit{A. sagrei}.\\

% The evidence for sexual selection

\paragraph{Sexual selection could be at play} Sexual selection is an important component of an adaptive, sensory drive hypothesis of dewlap color divergence across environmental conditions, because in this scenario habitat-dependent selection is executed by the signal recipients \citep{Endler1988}, such as mating partners \citep{Driessens2014}. An alternative yet non-mutually exclusive explanation for the observed idiosyncratic patterns of within-island divergence is that of "Fisherian" sexual selection \citep{Andersson1994}. Under the Fisherian model of sexual selection, arbitrary female preferences (i.e. independent of the environment) for certain male ornaments may drive divergent evolution of male signals, such as dewlap coloration, if female preferences differ between localities for other reasons than environment-dependent perception abilities (a situation that could have arisen e.g. in an initial phase of genetic drift). Substantial levels of promiscuity in \textit{A. sagrei} suggest ample opportunity for female mate choice \citep{Kamath2018}, and are in line with this explanation. Therefore, arbitrary, habitat-independent female preferences could further explain the inconsistent patterns of divergence across islands. In contrast with this, however, \citet{Baeckens2018} found no link between \textit{A. sagrei} dewlap coloration and size dimorphism (a proxy for sexual selection) in an among-island study of the same archipelagos.\\

Besides, although this process is well-established in some systems such as African cichlids (e.g. \citealt{Seehausen1997}), previous work suggests that dewlap coloration in \textit{Anolis} may act as a mating cue according to a "good genes" model of sexual selection \citep{Andersson1994}, rather than a Fisherian process. In the good genes model, the cue under sexual selection is an indicator of individual quality (e.g. better immune response to disease) and of indirect benefit to the offspring. For example, \citet{Cook2013} found lower orange reflectance in dewlaps with heavily parasitized \textit{A. brevirostris}, suggesting a trade-off in carotenoid use between the immune response and pigment deposition. \citet{Steffen2014} found that lower UV and orange-red reflectance predict contest-winning success between \textit{A. sagrei} males, while \citet{Driessens2015} further found that more yellow and red dewlaps (relative to UV) predict better body condition, and that higher yellow and UV reflectance at the margin of the dewlap predict higher hematocrit (the concentration of red blood cells), indicating a better health. Other aspects of the dewlap than color have also been found to be indicators of individual quality, such as dewlap size \citep{Vanhooydonck2005, Vanhooydonck2009}, but not dewlap display frequency \citep{Tokarz2002, Tokarz2005, Driessens2015}.\\

\paragraph{A role of phenotypic plasticity is unlikely} Differences in coloration between habitat populations may not be genetically determined, and may be influenced by environmental factors such as parasite load, as mentioned above \citep{Cook2013}. The yellow, orange and red coloration in anoline dewlaps are produced by pterins and carotenoids \citep{Ortiz1962, Ortiz1962a, Ortiz1963, Ortiz1966, Macedonia2000, Steffen2007, Steffen2009}. Animals lack the ability to synthesize carotenoids, and those must therefore be found in the diet, while pterins are synthesized from nucleotides \citep{Goodwin1984, Hill2002, Hill2006}. However, experimental manipulation of dietary carotenoid content showed no effect on dewlap coloration in \textit{A. sagrei} \citep{Steffen2010} nor in \textit{A. distichus} \citep{Ng2013}, which also has an orange-based dewlap. This makes a plastic response to differences in diet across habitats unlikely. Furthermore, developmental plasticity during the ontogeny is also unlikely because dewlap coloration develops at sexual maturity in anoles \citep{Ng2013}. The differences we observed could therefore be heritable. This hypothesis is further supported by \citet{Cox2017}, who found a high degree of heritability of dewlap coloration in \textit{A. sagrei}. Although most studies used one or two-generation common garden experiments and thus could not rule out transgenerational plastic effects \citep{Tariel2020}, dewlap coloration generally seems to not be a plastic trait. This further reinforces an adaptive explanation, where dewlap color could be under differential natural and/or sexual selection in these different habitats.\\

% Conclusion, and a word about speciation

\paragraph{Implications in the context of speciation} Local adaptation can be a precursor to ecological speciation, a process that may have given rise to the adaptive radiation of \textit{Anolis} lizards \citep{Harmon2003, Gavrilets2009}. Ecologically-mediated divergence of a sexual signal may be a potent path to the evolution of reproductive isolation through divergent sexual selection \citep{Reynolds2007, Servedio2011}. Evidence suggests that dewlap coloration could take this role in anoles \citep{Ng2011, Lambert2013, Geneva2015, Ng2017}, or at least that it is frequently involved in species recognition \citep{Williams1969, Williams1977, Losos1985, Macedonia1994, Fleishman2000, Macedonia2013, Ingram2016, Baeckens2018}. Although this signal is not detected at the phylogenetic scale of the whole genus \citep{Nicholson2007, Harrison2012, Ingram2016}, sexual signals are often evolutionarily very labile \citep{Kraaijeveld2011}, and the anole dewlap in particular is capable of rapid macroevolution; for example, \textit{A. conspersus} on Grand Cayman evolved a UV-blue dewlap from an ancestral orange dewlap in 2 to 3 million years \citep{Macedonia2001}. We present evidence of multiple cases of potentially adaptive maintenance of habitat-associated dewlap divergence over small geographical scale in \textit{A. sagrei} across the West Indies. While these intra-island populations do not appear to be in the process of speciation, our results suggest that the anoline dewlap has enough micro-scale, local adaptive potential to contribute of reproductive isolation, should it be recruited for assortative mating.