
% There are inconsistent differences that are not explained by distance

\textbf{Jonathan's take: too much push in favor of selection, without really having much evidence. Simple models of genetic drift are ruled out and we don’t think it is plasticity. However, our primary test for adaptation—looking for parallel responses across islands—also was falsified. We suggest, however, that selection could be involved, and suggest several ways this might have happened.}

\paragraph{Dewlap coloration differs between habitat-types} We found that male dewlap coloration in \textit{A. sagrei} significantly varied between fine-scale habitat-types (beach scrub bush, primary coppice forest and mangrove forest) on five islands of the West Indies: Abaco, Bimini, Cayman Brac, Little Cayman and Long Island. However, the habitat-specific variation in dewlaps was not consistent between these islands. Although those results are consistent with selection acting at a very local scale, other evolutionary drivers could be at work, such as phenotypic plasticity, random drift, or multiple colonization events. We reject this last explanation because all of the island populations in this study are strictly monophyletic, implying a single colonization event per island (van de Schoot, unpublished thesis; \citealt{Driessens2017, Reynolds2020}).

\paragraph{A role of neutral drift is unlikely} Differences in organismal traits between environments are not necessarily proof of adaptation or selection, and genetic drift may result in patterns similar to local adaptation \citep{Miles2019}. Nevertheless, two lines of evidence from our data suggest that this scenario may be implausible. First, we found little evidence for a role of phenotypic isolation-by-distance (spatial autocorrelation) in explaining the differences we report. We did detect a significant signal of isolation-by-distance on Eleuthera, but there were no differences in dewlap coloration between habitats on this island. Second, we detected differences between habitats at relatively small spatial scales, most of the time between sites 5-10km apart, sometimes a few hundred meters away, on Bimini for example. Such small-scale differences would be unlikely under strong gene flow \citep{Richardson2014}. Our study islands lack geographic barriers to the movement of \textit{A. sagrei}, which have been shown to be highly mobile \citep{Kamath2018}, implying widespread gene flow across sites and habitats. Moreover, habitat-populations within each island were found to be non-monophyletic and often share identical haplotypes, based on phylogenetic analysis of mitochondrial DNA sequences (van de Schoot et al. unpublished thesis), suggesting gene flow between habitats may be widespread. Our results align with previous documented cases of persistent dewlap color divergence despite gene flow in multiple species of anoles, sometimes in relation to environmental conditions. \citet{Ng2012} and \citet{Ng2016} found divergent dewlap coloration in the face of gene flow between subspecies of \textit{A. distichus} across Hispaniola, and proposed this as a mechanism of reproductive isolation in the early stages of speciation \citep{Ng2011, Lambert2013, Ng2017}. \citet{Stapley2011} found that dewlap color polymorphism was maintained in the absence of genetic structure between populations of \textit{A. apletophallus} from Panama. \citet{Thorpe2002a} found that divergence in dewlap coloration matched habitat-type better than mitochondrial lineage in \textit{A. roquet} on Martinique, and a convergent pattern was found in \textit{A. trinitatis} on the featureless island of St Vincent \citep{Thorpe2002b}. Finally, regionally-distinct body coloration, but not dewlap coloration, is present in \textit{A. conspersus} on another small island, Grand Cayman, where no physical barriers to gene flow exist \citep{Macedonia2001}.\\

% \paragraph{No conclusive evidence for local adaptation of dewlap coloration} Although phenotypic divergence at small spatial scales may be an indicator of local adaptation, the inconsistency of the between-habitat divergence patterns we observed across islands implies that neutral processes cannot be definitely ruled out \citep{Losos2011}. Conversely, the absence of parallel divergence across islands does not rule out local adaptation either \citep{Losos2011}. If selection has an effect, it may be dependent on aspects of the environment that are not encompassed by our broad habitat-type classification into coastal dry scrub, primary coppice forest and mangrove forest, and dewlap coloration may be influenced by a mosaic of interacting local environmental factors, which need not be the same across islands.\\

\paragraph{No conclusive evidence for local adaptation of dewlap coloration} One of the most informative tests for adaptation is the convergence of differentiation patterns across replicate islands or localities \citep{Losos2009, Losos2011}. Previous studies have described convergent patterns of dewlap color evolution in similar environments across islands and species \citep{Thorpe2002a, Thorpe2002b}. Others have suggested that dewlap coloration may have evolved to be maximally detectable under local light conditions imposed by the environment, primarily through UV contrast (i.e. UV-brighter dewlaps in UV-dark, mesic habitats and UV-darker dewlaps in UV-bright, xeric habitats), in \textit{A. cristatellus} and \textit{A. cooki} from Puerto Rico \citep{Leal2002, Leal2004}. Although UV reflectance was commonly involved in between-habitat divergence in \textit{A. cooki} and \textit{A. cristatellus}, we found no such patterns in \textit{A. sagrei}, where instead, we found the darkest dewlaps in the dark, mesic habitat -- primary coppice forest -- on three islands, and dewlaps often differed the most between beach scrub and mangrove forest, two xeric habitats with similar, high irradiance levels \citep{Howard1950, Schoener1968}. The inconsistent and idiosyncratic patterns we observed suggest that dewlap color variation between habitats cannot be predicted by habitat identity alone, and the test of convergence across islands is rejected. Studies of Jamaican and Hispaniolan anoles similarly found between-habitat differences in dewlap coloration but no evidence for higher dewlap detectability in different habitats \citep{Fleishman2009, Ng2012}. Our data are consistent with those previous results in suggesting that adaptation to local light conditions, or at least broad habitat types, is not a major driver of the within-island variation in dewlap coloration in \textit{A. sagrei}. Habitats on different islands may also differ in aspects other than light conditions, such as densities of predators or congeners, which have been shown to affect among-island dewlap diversity \citep{Vanhooydonck2009, Baeckens2018}. In particular, \citet{Baeckens2018} recently showed that dewlaps with spotted patterns occurred more often in \textit{A. sagrei} on islands with more coexisting species of anoles. Therefore, if local adaptation occurs, it is more likely to involve components of the environment that do not encompass our broad habitat categories.

\paragraph{Sexual selection could be at play} Selection, however, needs not necessarily be linked to habitat type, and may take the form of arbitrary, "Fisherian" sexual selection, where female preferences differ between localities for reasons unrelated to the environment, driving the divergent evolution of male ornaments \citep{Andersson1994}. This process could account for the idiosyncratic patterns of within-island divergence we report, where initial differences in female preferences could have arisen for nonselective reasons (e.g. plasticity or random drift). Substantial levels of promiscuity in \textit{A. sagrei} suggest ample opportunity for female mate choice \citep{Kamath2018}, and are in line with this scenario. However, \citet{Baeckens2018} found no link between \textit{A. sagrei} dewlap coloration and size dimorphism (a proxy for sexual selection) in an among-island study of the same archipelagos. Another form of sexual selection is the "good genes" model, where the cue under sexual selection is an indicator of individual quality (e.g. better immune response to disease) and of indirect benefit to the offspring \citep{Andersson1994}. Several studies suggest that this possibility could be the case for anoles. For example, \citet{Cook2013} found lower orange reflectance in dewlaps with heavily parasitized \textit{A. brevirostris}, suggesting a trade-off in carotenoid use between the immune response and pigment deposition. \citet{Steffen2014} found that lower UV and orange-red reflectance predict contest-winning success between \textit{A. sagrei} males, while \citet{Driessens2015} further found that more yellow and red dewlaps (relative to UV) predict better body condition, and that higher yellow and UV reflectance at the margin of the dewlap predict higher hematocrit (the concentration of red blood cells), indicating a better health. Other aspects of the dewlap than color have also been found to be indicators of individual quality, such as dewlap size \citep{Vanhooydonck2005, Vanhooydonck2009}, but not dewlap display frequency \citep{Tokarz2002, Tokarz2005, Driessens2015}. If dewlap coloration is indeed an indicator of male quality in \textit{A. sagrei}, however, this would explain the observed patterns only if cues associated with good genes differ between sites, which is unlikely as it implies an association of male fitness with different dewlap coloration in different habitats, the evolution of which may be counteracted by gene flow (citation to say that if gene flow counters the divergence of one trait, then it is even worse for the association of two traits). Basically, one would expect good genes-sexual selection to favor the same coloration across the archipelago, and would therefore not result in local differences between habitats.


% An alternative yet non-mutually exclusive explanation for the observed idiosyncratic patterns of within-island divergence is that of "Fisherian" sexual selection \citep{Andersson1994}. 

% Under the Fisherian model of sexual selection, arbitrary female preferences (i.e. independent of the environment) for certain male ornaments may drive divergent evolution of male signals, such as dewlap coloration, if female preferences differ between localities for other reasons than environment-dependent perception abilities (a situation that could have arisen e.g. in an initial phase of genetic drift). Substantial levels of promiscuity in \textit{A. sagrei} suggest ample opportunity for female mate choice \citep{Kamath2018}, and are in line with this explanation. Therefore, arbitrary, habitat-independent female preferences could further explain the inconsistent patterns of divergence across islands. In contrast with this, however, \citet{Baeckens2018} found no link between \textit{A. sagrei} dewlap coloration and size dimorphism (a proxy for sexual selection) in an among-island study of the same archipelagos.\\

% Alternatively, dewlap coloration in \textit{Anolis} may act as a mating cue according to a "good genes" model of sexual selection, rather than a Fisherian process. In the good genes model, the cue under sexual selection is an indicator of individual quality (e.g. better immune response to disease) and of indirect benefit to the offspring \citep{Andersson1994}. Several studies suggest that this possibility could be the case for anoles. For example, \citet{Cook2013} found lower orange reflectance in dewlaps with heavily parasitized \textit{A. brevirostris}, suggesting a trade-off in carotenoid use between the immune response and pigment deposition. \citet{Steffen2014} found that lower UV and orange-red reflectance predict contest-winning success between \textit{A. sagrei} males, while \citet{Driessens2015} further found that more yellow and red dewlaps (relative to UV) predict better body condition, and that higher yellow and UV reflectance at the margin of the dewlap predict higher hematocrit (the concentration of red blood cells), indicating a better health. Other aspects of the dewlap than color have also been found to be indicators of individual quality, such as dewlap size \citep{Vanhooydonck2005, Vanhooydonck2009}, but not dewlap display frequency \citep{Tokarz2002, Tokarz2005, Driessens2015}.\\

\paragraph{A role of phenotypic plasticity is unlikely} Differences in coloration between habitat populations may not be genetically determined and instead may be influenced by environmental factors such as parasite load, as mentioned above \citep{Cook2013}. The yellow, orange and red coloration in anoline dewlaps are produced by pterins and carotenoids \citep{Ortiz1962, Ortiz1962a, Ortiz1963, Ortiz1966, Macedonia2000, Steffen2007, Steffen2009}. Animals lack the ability to synthesize carotenoids, and those must therefore be found in the diet, while pterins are synthesized from nucleotides \citep{Goodwin1984, Hill2002, Hill2006}. However, experimental manipulation of dietary carotenoid content showed no effect on dewlap coloration in \textit{A. sagrei} \citep{Steffen2010} nor in \textit{A. distichus} \citep{Ng2013}, which also has an orange-based dewlap. This makes a plastic response to differences in diet across habitats unlikely. Furthermore, developmental plasticity during the ontogeny is also unlikely because dewlap coloration develops at sexual maturity in anoles \citep{Ng2013}. The differences we observed could therefore be genetically based. This hypothesis is further supported by \citet{Cox2017}, who found a high degree of heritability of dewlap coloration in \textit{A. sagrei}. Response to parasite load is definitely an example of plasticity!\\ % playing it safer here by not saying it applies to all anoles, but also not mentioning the transgenerational effects anymore

% For these reasons, we consider implausible that dewlap coloration is a plastic trait in our study species.\\

% Too broad a statemtent to make
% Although most studies used one or two-generation common garden experiments and thus could not rule out transgenerational plastic effects \citep{Tariel2020}, dewlap coloration generally seems to not be a plastic trait.

% This further reinforces an adaptive explanation, where dewlap color could be under differential natural and/or sexual selection in these different habitats.

% Conclusion, and a word about speciation

\paragraph{Implications in the context of speciation} Local adaptation can be a precursor to ecological speciation, a process that may have given rise to the adaptive radiation of \textit{Anolis} lizards \citep{Harmon2003, Gavrilets2009}. Ecologically-mediated divergence of a sexual signal may be a potent path to the evolution of reproductive isolation through divergent sexual selection \citep{Reynolds2007, Servedio2011}. Evidence suggests that dewlap coloration could take this role in anoles \citep{Ng2011, Lambert2013, Geneva2015, Ng2017}, or at least that it is frequently involved in species recognition \citep{Williams1969, Williams1977, Losos1985, Macedonia1994, Fleishman2000, Macedonia2013, Ingram2016, Baeckens2018}. Although a correlation between dewlap coloration and reproductive isolation is not detected at the phylogenetic scale of the whole genus \citep{Nicholson2007, Harrison2012, Ingram2016}, sexual signals are often evolutionarily very labile \citep{Kraaijeveld2011}, and the anole dewlap in particular is capable of rapid macroevolution throughout its macroevolutionary history \citep{Nicholson2007}. For example, \textit{A. conspersus} on Grand Cayman evolved a UV-blue dewlap from an ancestral orange dewlap in 2 to 3 million years \citep{Macedonia2001}. We present evidence of multiple cases of within-island dewlap divergence over small geographical scales in \textit{A. sagrei} across the West Indies. While these intra-island populations do not appear to be in the process of speciation, our results suggest that the anoline dewlap has enough micro-scale, local adaptive potential to contribute to reproductive isolation, should it be recruited for assortative mating.