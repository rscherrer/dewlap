% Citation for classifiers as a good group comparison tool
% Yang and Yang on LDA being performed on PC

\subsection*{Data collection}
	
We sampled 466 lizards from seven islands in the Bahamas Archipelago -- Abaco, North Andros, South Andros, South Bimini, Eleuthera, Long Island, Ragged Island -- and two in the Cayman Islands -- Cayman Brac and Little Cayman (Figure \ref{fig:map}). These islands and island banks were chosen to span the West Indian range of \textit{Anolis sagrei}. Three habitats were sampled on each island based on characterizations by \citet{Howard1950} and \citet{Schoener1968}. Each habitat is clearly distinguishable by their dominant vegetation type --- xeric coastal scrub (open, relatively dry habitat consisting of low vegetation or isolated trees), primary coppice forest (closed-canopy forest) and mangrove forest (wet coastal habitat with trees growing in brackish water and high light penetration). Sample sizes are given in Table \ref{suptab:counts}. Our sampling design enabled us to test for differences between habitats at a coarse and fine geographical scale. The median distance between two localities within an island was $11.18$km, with some islands being sampled at smaller or larger scales (Figure \ref{supfig:map}, Table \ref{suptab:locations}). $80.3$\% of all pairwise distances within islands were below $50$km. Additionally, there are no major barriers to dispersal (such as mountains) on any of the islands that we sampled.
	
\subsection*{Reflectance measurements}

We measured reflectance between 300 and 700nm wavelength, a range that encompasses the colors visible to most lizards and vertebrates in general \citep{Lazareva2012}. Measurements were taken with an Ocean Optics USB4000 spectrometer, a pulsed Xenon light source (PX-2, Ocean Optics, Largo, FL, USA) and a reflectance probe protected by a black anodized aluminum sheath. Measurements were taken with a 45-degree inclination to prevent specular reflection \citep{Endler1990}. The device was regularly standardized with a Spectralon white standard (Labsphere, North Sutton, NH, USA). Reflectance was measured at the center of the dewlap.

\subsection*{Analysis}

All analyses in this study were performed in R 3.6.1 \citep{RCoreTeam2019}. 

\subsubsection*{Dimensionality reduction}

Reflectance curves were smoothened using the R package pavo \citep{Maia2013} as well as custom R functions (where are those?) to results into one reflectance value at each nanometer in wavelength from 300 to 700nm. Because neighboring wavelengths are highly collinear in reflectance, we reduced the dimensionality of the data using principal component analysis (PCA), as per \citet{Cuthill1999, Leal2002}. We performed PCA on each island separately and systematically retained the first four principal components (PC), which together always explained more than $88.8\%$ of the variance across islands (Table \ref{suptab:pcavariances}). PC1 explained between $40$ and $56$\% of the variance across islands; PC2 explained $17.4$--$27.9$\%; PC3 $12.7$--$17.6$\% and PC4 $4.3$--$10.5$\%. The first four PCs explained similar proportions of variance when calculated for all islands together (Table \ref{suptab:pcavariances}). PCs need not represent the same wavelengths across islands because they are fitted on different datasets. Nevertheless, PC1 was very collinear with brightness for all islands (Table \ref{suptab:brightness}). PC2 correlated highly with the red and ultraviolet ends of the spectrum, which were inversely correlated with each other (Fig. \ref{fig:boxplots}A). Higher PCs corresponded to various combinations of wavelengths. Because PC1 correlated uniformly with all wavelengths across the spectrum  we considered PC2 onwards to capture the chromatic dimensions of color space, i.e. the relative contributions of the wavelengths regardless of brightness.

\subsubsection*{Machine learning}

Our data violated the multivariate analysis of variance (MANOVA) assumption of homogeneity of covariance matrices across groups for all islands but Ragged Island (Box's M-test, \citealt{Box1949, Morrison1988}, implemented in the R package heplots, \citealt{Fox2018},  Table \ref{suptab:covariances}). We also detected within-habitat deviations from multivariate normality, primarily on Abaco, Bimini and Eleuthera (Henze-Zirkler's test, \citealt{Henze1990}, implemented in the R package MVN, \citealt{Korkmaz2014}, Table \ref{suptab:multinorm}). For these reasons and to reduce the chances of false discovery, we conducted multivariate group comparisons using support vector machines (SVMs), a model-free, nonparametric supervised machine learning technique.\\

% Box M is sensitive to non-normality \citep{Fox2018}

Machine learning for group comparison has become more popular in ecology and evolution in the recent years (e.g. \citet{Pigot2020}). In particular, SVMs are designed to find the best possible nonlinear boundaries between labelled groups of points in multidimensional spaces, without assumptions about the distribution of the data \citep{Cortes1995, Cristianini2000, Kim2018}. This makes them well suited to field biological data, which often violate the assumptions of classical linear modeling \citep{Kim2018} and can be, as in the case of coloration, inherently highly multivariate \citep{Cuthill1999}. The significance and robustness of a classification is assessed by training a machine to recognize differences between groups on part of the data, and using that fitted machine to predict the group-labels of data points from a testing set that were not included in the training, based solely on their multivariate coordinates. This cross-validation procedure results in a proportion of correctly classified points, or generalization accuracy score, which can be compared to that expected under random guessing using a binomial test.\\

In this study, we performed SVM classifications on each island separately. We used a standard five-fold cross-validation procedure, where the data were randomly split into five bins of approximately equal sizes, and each bin was in turn taken as the testing set while the rest was used as a training set, thus resulting in five trained machines per cross-validation. We replicated this procedure 100 times for each island to account for stochastic outcomes. We performed binomial tests to evaluate the significance of the deviations in observed mean generalization accuracy per island to null expectations under random guessing. Each training data set was downsampled to the size of its least represented habitat to ensure balanced training samples. We ensured that each habitat was represented by at least five data points in the training set. All classification analyses were repeated using the more classical linear discriminant analysis (LDA), a supervised machine learning technique finding linear boundaries that maximize the differences between groups, albeit assuming multivariate normality and homogeneity of covariance matrices \citep{Ripley1996a}. We used the R package rminer \citep{Cortez2010, Cortez2016} for SVMs, and MASS (citation) for LDAs. We used rminer's default heuristic search option to automatically tune the Gaussian kernel parameter $\sigma$ and the complexity parameter $C$ for the SVMs.\\

\subsubsection*{Univariate analyses}

For each island where significant differences in multivariate dewlap coloration were detected between habitats, we used multiple univariate analyses of variance (ANOVA) to identify which variables were responsible for the observed differences. We constructed our ANOVA models in two steps, as per \citet{Zuur2009}. In a first step, we accounted for heterogeneity of variances across groups by systematically comparing the goodness-of-fit of an ANOVA model estimated with ordinary least squares (OLS) with that of a model estimated with generalized least squares (GLS), which allowed one estimate of residual variance per habitat (using the R package nlme, \citealt{Pinheiro2000, Pinheiro2020}). Both models were fitted with restricted maximum likelihood (REML). Goodness-of-fit was estimated using Akaike's Information Criterion corrected for small sample sizes (AICc, from the R package MuMIn, \citealt{Barton2019}), and the estimation method yielding the lowest AICc was retained. In a second step, we re-fitted the retained model with maximum likelihood (ML) to test for the effect of habitat-type using likelihood ratio tests (LRT) between a model including a habitat-term and a null model lacking it. Significant \textit{post hoc} contrasts were assessed using Tukey's Honest Significant Difference (HSD) test whenever the OLS-model was the best fit, and pairwise Wilcoxon tests whenever there was evidence for heterogeneity of variances, and the GLS-model was the best fit.\\

We further tested the robustness of our results to non-normality. For each habitat-island combination where multivariate non-normality was detected (Table \ref{suptab:multinorm}), we assessed the univariate normality of each variable (Shapiro-Wilk's test, Table \ref{suptab:normality}) and backed our ANOVA results with nonparametric Kruskal-Wallis tests every time significant deviations were detected.\\

\subsubsection*{Spatial autocorrelation}

We tested for within-island spatial autocorrelation between the geographical distances among sampling sites and their euclidean distances in multivariate color space (mean PC1 to PC4 per site, Table \ref{suptab:locations}), regardless of habitat-type. Because often only a few sites were sampled per island, we could not get meaningful results from tests using sites as units of observation. Instead, we designed a permutation test where we randomly reshuffled individual lizards across sites within islands 1,000 times each, and systematically recalculated Pearson's correlation coefficient between geographic distances (computed as geodesic distances using the R package geosphere, \citealt{Hijmans2019}) and phenotypic distances. We used the resulting null distributions of correlation coefficients to assess the significance of the observed spatial autocorrelation for each island (Table \ref{suptab:autocor}).

