% Citation for classifiers as a good group comparison tool
% Yang and Yang on LDA being performed on PC

\subsection*{Data collection}
	
We sampled 466 male \textit{Anolis sagrei} from seven islands in the Bahamas Archipelago -- Abaco, North Andros, South Andros, South Bimini, Eleuthera, Long Island, Ragged Island -- and two in the Cayman Islands -- Cayman Brac and Little Cayman (Figure \ref{fig:map}). These islands and island banks were chosen to span the breadth of the West Indian range of \textit{Anolis sagrei}, because they have highly similar habitat types,  and because the \textit{A. sagrei} on each island group are derived from ancient and distinct colonization events from Cuba (i.e. relatively evolutionarily independent, \citealt{Reynolds2020}). Three habitats were sampled on each island based on characterizations by \citet{Howard1950} and \citet{Schoener1968}. Each habitat is clearly distinguishable by their dominant vegetation type --- xeric coastal scrub (open, relatively dry habitat consisting of low vegetation or isolated trees), primary coppice forest (closed-canopy forest) and mangrove forest (wet coastal habitat with trees growing in brackish water and high light penetration). Sample sizes are given in Table \ref{suptab:counts}. Our sampling design enabled us to test for differences between habitats at a coarse and fine geographical scale. The median distance between two localities within an island was $11.18$km, with some islands being sampled at smaller or larger scales (Figure \ref{supfig:map}, Table \ref{suptab:sites}). $80.3$\% of all pairwise distances within islands were less than $50$km. Additionally, there are no major barriers to dispersal (such as mountains or grassland) on any of the islands that we sampled.
	
\subsection*{Reflectance measurements}

We measured reflectance between 300 and 700nm wavelength, a range that encompasses the colors visible to most lizards and vertebrates in general \citep{Lazareva2012}. Measurements were taken with an Ocean Optics USB4000 spectrometer, a pulsed Xenon light source (PX-2, Ocean Optics, Largo, FL, USA) and a reflectance probe protected by a black anodized aluminum sheath. Measurements were taken with a 45-degree inclination to prevent specular reflection \citep{Endler1990}. The device was regularly standardized with a Spectralon white standard (Labsphere, North Sutton, NH, USA). Reflectance was measured at the center of the dewlap.

\subsection*{Analysis}

All analyses in this study were performed in R 3.6.1 \citep{RCoreTeam2019}. 

\subsubsection*{Dimensionality reduction}

Reflectance curves were smoothed using the R package pavo \citep{Maia2013} as well as with custom R functions, down to one reflectance value at each nanometer in wavelength from 300 to 700nm. Because neighboring wavelengths are highly collinear in reflectance, we reduced the dimensionality of the data using principal component analysis (PCA), as per \citet{Cuthill1999} and \citet{Leal2002}. We performed PCA on each island separately and systematically retained the first four principal components (PC), which together always explained more than $88.8\%$ of the variance across islands (Table \ref{suptab:pcavariances}). PC1 explained between $40$ and $56$\% of the variance across islands; PC2 explained $17.4$--$27.9$\%; PC3 $12.7$--$17.6$\% and PC4 $4.3$--$10.5$\%. The first four PCs explained similar proportions of variance when calculated for all islands together (Table \ref{suptab:pcavariances}). PCs need not represent the same wavelengths across islands because they are fitted on different datasets. Nevertheless, PC1 was very collinear with brightness for all islands (Figure \ref{supfig:brightness}, Table \ref{suptab:brightness}). PC2 correlated highly with the red and ultraviolet ends of the spectrum, which were inversely correlated with each other (Fig. \ref{fig:anova}A). Higher PCs corresponded to various combinations of wavelengths. Because PC1 correlated uniformly with all wavelengths across the spectrum  we considered PC2 onwards to capture the chromatic dimensions of color space, i.e. the relative contributions of the wavelengths regardless of brightness.

\subsubsection*{Pooled analyses}

In addition to within-island PCA, we performed a PCA on pooled data from the whole archipelago. The first four principal components explained 91.3\% of the variance (Table \ref{suptab:pcavariances}). Again PC1 strongly correlated with brightness (Fig. \ref{supfig:brightness_pooled}, Table \ref{suptab:brightness}). PC2 was positively correlated to short wavelengths (ultraviolet to blue) and negatively correlated to long wavelengths (green to red, Fig. \ref{supfig:pooled}B). PC3 was strongly negatively correlated with UV reflectance and positively correlated with blue-green. PC4 was made of a mosaic of wavelengths, correlating positively with blue and red but negatively with ultraviolet and yellow.\\

We used this dataset to partition the variance in dewlap coloration among islands, habitats and habitats within islands, using a two-way multivariate analysis of variance (MANOVA) with an interaction term. However, because the assumptions of parametric MANOVA were violated for all islands but Ragged Island (multivariate normality, Henze-Zirkler's test, \citealt{Henze1990}, R package MVN, \citealt{Korkmaz2014}, Table \ref{suptab:multinorm}; and homogeneity of covariance matrices, Box's M-test, \citealt{Box1949, Morrison1988}, R package heplots, \citealt{Fox2018}, Table \ref{suptab:covariance}), we used a semi-parametric MANOVA instead (R package MANOVA.RM, \citealt{Friedrich2018}), with P-values calculated from a bootstrap procedure with 1,000 iterations. We calculated the proportion of variance explained by islands, habitats and the habitat-by-island interaction using partial effect sizes $\eta^2$ on a MANOVA-approximation of the analysis (R package heplots, \citealt{Fox2018}).

\subsubsection*{Machine learning}

% move this to the previous section?

Because of the aforementioned violations of the MANOVA assumptions, and to reduce the chances of false discovery, we conducted multivariate group comparisons using support vector machines (SVMs), a model-free, powerful nonparametric supervised machine learning technique.\\

% Box M is sensitive to non-normality \citep{Fox2018}

Machine learning for group comparison has become more common in ecology and evolution in recent years (e.g. \citealt{Pigot2020}). In particular, SVMs are designed to find the best possible nonlinear boundaries between labelled groups of points in multidimensional spaces, without assumptions about the distribution of the data \citep{Cortes1995, Cristianini2000, Kim2018}. This makes them well suited to field biological data, which often violate the assumptions of classical linear modeling \citep{Kim2018} and can be, as in the case of coloration, inherently highly multivariate \citep{Cuthill1999}. First, a machine is trained to recognize differences between groups within a subset of the data called the training set. Significance of differences is then assessed by testing the accuracy of that fitted machine in predicting the group-labels of data points that were not included in the training, called a testing set, based solely on their multivariate coordinates. This cross-validation procedure results in a proportion of correctly classified points, or generalization accuracy score, which can be compared to that expected under random guessing using a binomial test.\\

In this study, we performed SVM classifications on each island separately. We used a standard five-fold cross-validation procedure, where the data were randomly split into five bins of approximately equal sizes. Each bin was in turn taken as the testing set while the rest was used as a training set, thus resulting in five trained machines per cross-validation. We replicated this procedure 100 times for each island to account for stochastic outcomes. We performed binomial tests to evaluate the significance of deviations in observed mean generalization accuracy per island to null expectations under random guessing. Each training data set was downsampled to the size of its least represented habitat to ensure balanced training samples. We ensured that each habitat was represented by at least five data points in the training set.\\ 

All classification analyses were repeated using the more classical linear discriminant analysis (LDA), a supervised machine learning technique finding linear boundaries that maximize the differences between groups, albeit assuming multivariate normality and homogeneity of covariance matrices \citep{Ripley1996a}. We used the R package rminer \citep{Cortez2010, Cortez2016} for SVMs, and MASS \citep{Venables2002} for LDAs. We used rminer's default heuristic search option to automatically tune the Gaussian kernel parameter $\sigma$ and the complexity parameter $C$ for the SVMs.\\

The same procedure was repeated on principal components from the whole archipelago (see Pooled analyses) to evaluate the significance of archipelago-wide differences in dewlap coloration across habitats.\\ 

All machine learning classifications performed on principal components were also repeated on the original reflectance datasets reduced to 50-nm spaced wavelengths from 300 to 700nm.\\

We conducted one-dimensional sensitivity analyses using rminer \citep{Cortez2013} to determine the relative importance of the different input variables during classification where significant differences were detected, both on machines trained on principal components and machines trained on non-transformed reflectance at various wavelengths. In parallel, we conducted univariate analyses of variance to independently test the importance of different variables in between-habitat variation, on islands where the machines detected significant differences based on binomial tests (next section).

\subsubsection*{Univariate analyses}

For each island where significant differences in multivariate dewlap coloration were detected between habitats, we used multiple univariate analyses of variance (ANOVA) to identify which variables were responsible for the observed differences. We constructed our ANOVA models in two steps, as per \citet{Zuur2009}. In a first step, we accounted for heterogeneity of variances across groups by systematically comparing the goodness-of-fit of an ANOVA model estimated with ordinary least squares (OLS) with that of a model estimated with generalized least squares (GLS), which allowed one estimate of residual variance per habitat (using the R package nlme, \citealt{Pinheiro2000, Pinheiro2020}). Both models were fitted with restricted maximum likelihood (REML). Goodness-of-fit was estimated using Akaike's Information Criterion corrected for small sample sizes (AICc, R package MuMIn, \citealt{Barton2019}), and the estimation method yielding the lowest AICc was retained. In a second step, we re-fitted the retained model with maximum likelihood (ML) to test for the effect of habitat-type using likelihood ratio tests (LRT) between a model including a habitat-term and a null model lacking the habitat-term.\\

We tested the assumptions of the parametric ANOVA for each island included in the univariate analyses. For all islands where deviations from multivariate normality were detected in at least one habitat (Table \ref{suptab:multinorm}), we assessed univariate normality for each principal component (Shapiro-Wilk's test, Table \ref{suptab:normality}). For skewed PCs that deviated significantly from normality, we repeated the analysis using a nonparametric Kruskal-Wallis test \citep{Hollander2013}. We found no multivariate outliers based on the Mahalanobis distance (package MVN, \citealt{Korkmaz2014}). We used the cases of better fit of the GLS model relative to the OLS model as evidence for heterogeneity of variances, which were then accounted for by the GLS approach (Table \ref{tab:anova}).\\

Significant \textit{post hoc} contrasts were assessed using Tukey's Honest Significant Difference (HSD) test whenever the assumptions of normality and homogeneity of variances was met \citep{Tukey1949}, Dunnett's T3 method when only homogeneity of variances was violated but not normality \citep{Dunnett1980}, and Nemenyi's test when normality was violated \citep{Nemenyi1963}. All \textit{post hoc} tests were performed with the R package PMCMRplus \citep{Pohlert2020}.\\

We used the same procedure to investigate which variables, if any, were involved in archipelago-wide multivariate differences between habitats detected in our two-way MANOVA design (see Pooled analyses). However, in the first step or our model comparison procedure, we added mixed-effect equivalents of our OLS and GLS models, this time with island as a random effect. The resulting four models were compared and the best fitting variance structure was retained as explained above.

\subsubsection*{Spatial autocorrelation}

We tested for within-island spatial autocorrelation between the geographical distances among sampling sites and their Euclidean distances in multivariate color space (mean PC1 to PC4 per site, Table \ref{suptab:sites}), regardless of habitat-type. Because often only a few sites were sampled per island, we could not get meaningful results from tests that use sites as units of observation, such as Moran's I test \citep{Gittleman1990}. Instead, we designed a permutation test where we randomly reshuffled individual lizards across sites within islands 1,000 times each, and systematically recalculated Pearson's correlation coefficient between geographic distances (computed as geodesic distances in the R package geosphere; \citealt{Hijmans2019}) and phenotypic distances. We used the resulting null distributions of correlation coefficients to assess the significance of the observed spatial autocorrelation for each island.

\subsubsection*{Site differences}

In this study, we were interested in the minimum spatial scale at which significant differences between habitats could be detected within islands. We performed multiple pairwise nonparametric Wilcoxon-Mann-Whitney tests \citep{Hollander2013} to compare dewlap coloration between sites with different habitat-types, for each pair of habitats and each variable where significant differences were detected with our analyses of variances. The P-values were adjusted using a Benjamini-Hochberg correction for multiple testing \citep{Benjamini1995}.