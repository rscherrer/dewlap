Here we describe more precisely the patterns identified on each island.\\

On Abaco, dewlaps from the mangrove habitat were the best discriminated, while dewlaps from the beach scrub habitat were often mistaken for dewlaps from the coppice habitat (Fig. \ref{fig:Abaco}D). Importance analysis revealed that beach scrub and mangrove lizards mostly differed in reflectance in the ultraviolet (UV) end of the spectrum (below 400nm, Fig. \ref{fig:Abaco_supplement}F), where mangrove dewlaps had higher UV reflectance relative to beach scrub lizards, and coppice lizards had an intermediate UV reflectance between the two other habitats (Fig. \ref{fig:Abaco}B). Consistent with this, our analyses of variance detected significantly higher PC2 scores in mangrove lizards than in the two other habitats (Fig. \ref{fig:Abaco}E, Table \ref{tab:anova}), representing a higher UV-reflectance relative to red (Fig. \ref{fig:Abaco}C). Beach scrub lizards also scored higher on PC3 (Fig. \ref{fig:Abaco}E, Table \ref{tab:anova}), indicating less curvature of the reflectance profile and relatively higher reflectance at intermediate wavelengths (blue-to-yellow) than at the ends of the range (Fig. \ref{fig:Abaco}C). Differences were detected between sites both at large ($\sim$ 100km) and short ($<$ 1km) distances (Fig. \ref{fig:Abaco_supplement}G). Abaco was the only island where we detected significant spatial autocorrelation (Table \ref{tab:autocorrelation}), that is, sites that were closer geographically tended to have populations of lizards with more similar dewlap colors.\\

On Bimini, the random forests mostly correctly classified lizards from the coppice and mangrove habitats while often misclassifying lizards from the beach scrub habitat (Fig. \ref{fig:Bimini}D). Relatively flat importance profiles for beach scrub lizards suggested that brightness was used instead of a particular wavelength to identify some of the beach scrub dewlaps (Fig. \ref{fig:Bimini}F). Indeed, some beach scrub dewlaps were substantially brighter than the rest (Fig. \ref{fig:Bimini}B, C), a pattern that was captured by our analysis of variance along PC1 (i.e. brightness, Fig. \ref{fig:Bimini}C, E, Table \ref{tab:anova}). Coppice dewlaps had significantly higher PC2 scores than mangrove dewlaps (Fig. \ref{fig:Bimini}E), suggesting a higher curvature (higher UV and red reflectance than more intermediate wavelengths, Fig. \ref{fig:Bimini}C). For these two habitats the random forests were most sensitive to UV reflectance (Fig. \ref{fig:Bimini}F). Beach scrub dewlaps had higher PC3 scores than coppice dewlaps but it was not clear what properties of spectral shape this principal component mapped onto (Fig. \ref{fig:Bimini}C). On this island, the beach scrub and coppice habitats were separated by a few hundred meters, making this contrast the smallest geographical scale at which differences in coloration were found in our study (Fig. \ref{fig:Bimini}G).\\

On Cayman Brac, all three habitats could be well discriminated against each other (Fig. \ref{fig:CaymanBrac}D), with UV reflectance appearing to be an important variable differentiating beach scrub and mangrove dewlaps (Fig. \ref{fig:CaymanBrac}F). In contrast, coppice dewlaps had a relatively flat importance profile, suggesting that brightness made them more distinct rather than any particular wavelength (Fig. \ref{fig:CaymanBrac}F). Consistent with this, coppice dewlaps were significantly different from all other dewlaps along PC1 (Fig. \ref{fig:CaymanBrac}E, Table \ref{tab:anova}). At a distance between 2 and 3km (Fig. \ref{fig:CaymanBrac}G), dewlaps in the beach scrub habitat reflected more red light (as represented by PC2, Fig. \ref{fig:CaymanBrac}C, E) and more UV (as represented by PC3, along which coppice dewlaps were intermediate, Fig. \ref{fig:CaymanBrac}C, E) than in the mangrove habitat.\\

On Eleuthera, although random forests detected between-habitat differences in dewlap color, other approaches did not (Tables \ref{tab:ldas} and \ref{tab:ksvms}), suggesting that the differences may be small. On Eleuthera, beach scrub and mangrove dewlaps were guessed relatively correctly, but coppice dewlaps were more often mistaken for mangrove dewlaps (Fig. \ref{fig:Eleuthera}D). Mangrove dewlaps had lower PC2 scores than beach scrub dewlaps (Fig. \ref{fig:Eleuthera}E, indicating higher UV relative to red, Fig. \ref{fig:Eleuthera}C), and higher PC4 scores than coppice dewlaps (Fig. \ref{fig:Eleuthera}E, suggesting more reflectance profiles with more curvature, Fig. \ref{fig:Eleuthera}C). Random forests did not seem to consistently capture the wavelengths responsible for these differences (Fig. \ref{fig:Eleuthera}F).\\

Little Cayman was characterized by a better discrimination of mangrove lizards from the rest than between beach scrub and coppice lizards even though all habitats were relatively well discriminated (Fig. \ref{fig:LittleCayman}D). Mangrove dewlaps were most distinct with respect to their reflectance in short wavelengths (Fig. \ref{fig:LittleCayman}F), with significantly lower UV reflectance (as represented by PC2, Fig. \ref{fig:LittleCayman}C, E, Table \ref{tab:anova}). Beach scrub lizards were characterized by brighter dewlaps than coppice lizards (PC1), and also more convex curves, i.e. slightly higher UV and red reflectance (as represented by higher PC3 scores), than lizards from the other two habitats (Fig. \ref{fig:LittleCayman}C, E, Table \ref{tab:anova}).\\

On Long Island the three habitats were relatively well discriminated (Fig. \ref{fig:LongIsland}D). Importance profiles indicated that short wavelengths were used to discriminate between beach scrub and mangrove lizards (Fig. \ref{fig:LongIsland}F). Beach scrub lizards had more curved reflectance profiles than in the mangrove, with higher levels of UV and red reflectance relative to intermediate wavelengths (PC3, Fig. \ref{fig:LongIsland}C, E, Table \ref{tab:anova}). Coppice lizards were significantly darker than mangrove and beach scrub lizards (PC1, Fig. \ref{fig:LongIsland}C, E, Table \ref{tab:anova}).\\ 

On North Andros beach scrub and coppice dewlaps could be discriminated better against each other than with mangrove dewlaps (Fig. \ref{fig:NorthAndros}D), with importance profiles supporting UV-reflectance as a predictor of coppice lizards (Fig. \ref{fig:NorthAndros}F). Coppice lizards had less curved reflectance profiles than beach scrub and mangrove lizards (PC2), and beach scrub dewlaps had the lowest scores on PC4, which was difficult to interpret (Fig. \ref{fig:NorthAndros}C, E, Table \ref{tab:anova}).\\

Classification success was not significantly better than expected by chance on Ragged Island and South Andros (Table \ref{tab:randomforests}, \ref{tab:ldas}, \ref{tab:ksvms}) where nearly no habitat could be differentiated from any other based on reflectance.