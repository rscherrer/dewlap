\documentclass{article}

\usepackage[english]{babel}
\usepackage[utf8x]{inputenc}
\usepackage[T1]{fontenc}
\usepackage[a4paper,top=3cm,bottom=2cm,left=3cm,right=3cm,marginparwidth=1.75cm]{geometry}
\usepackage[round]{natbib}
\usepackage{ulem}
\usepackage{xcolor}

\title{Dewlap color variation in \textit{Anolis sagrei} is maintained among habitats within islands of the West Indies}

\date{}

\begin{document}
	
	\maketitle
	
	\textbf{Object: Response letter}\\\\
	
	Dear Editor,\\
	
	Thank you for your feedback. In this response we address the comments we received for our submitted manuscript. Comments are numbered and quoted in \textit{italics}, and our responses follow in \textbf{bold}. Please find at the end a list of references mentioned in this letter, including new references that have been added to the the manuscript. For each response we indicate where in the manuscript the comment has been addressed. Please note that the numbers refer to the lines in the revised version of the manuscript, with changes tracked (deletions are \sout{crossed out} and additions are in \textcolor{olive}{olive}).\\
	
	In addition to responses to the reviewers' comments, we made a few additional edits to match the formatting requirements of \textit{Journal of Evolutionary Biology}: we placed each table on a separate page, and we added the taxonomic order and family of our study species upon first occurrence in the main text (l. 56).\\
	
	We look forward to hearing from you.\\\\
	
	Kind regards,\\
	
	Rapha\"{e}l Scherrer
	
	\pagebreak
	
	\subsection*{Editor}
	
	\subsubsection*{Comment 1}
	
	\textit{While I agree with reviewer 2 that this is a very nice paper with interesting results that could possibly be published as it stands, I also agree with reviewer 1 that an obvious ommision is any analysis of if or how the light environments differ between habitats and islands. How similar are the habitats between islands or could differences in habitat between islands explain some of the variation between islands? Including data and analysis on the differences in light environment between the habitats (and islands), and possibly doing some visual modelling (which I think would require information on the light environment), would move this from being a very good to an excellent paper. I would therefore like to give you to opportunity to consider these suggestions and whether you have or can reasonably obtain the data to do these analyses.}\\
	
	\textbf{Thank you for this suggestion. We agree with the Editor that this would be a great addition. Unfortunately, we are unable to obtain these data. We do not have light environment data for all our habitats and islands. But, we do have some preliminary data from a single island that show that light environments are different. It is fairly obvious to the observer that light environment differs, say, from coppice forest to coastal. In a newly added paragraph (lines 456-467), we discuss the possibility that differences in habitats between islands may influence the patterns of phenotypic variation.
	As for visual modeling, this work has not been done such that we could compare \textit{sagrei} with other anoles and their predators, but this could certainly be an interesting future avenue. Please also see our response to Comment 3.}

	\subsubsection*{Comment 2}
	
	\textit{One other more minor point is that lack of evidence for spatial autocorrelation is not the same as evidence for a lack of spatial autocorrelation (indeed there is a reasonably strong, albeit not quite "significant" spatial correlation on Eleuthera), so some of the conclusions about a lack of isolation by distance and genetic drift should be toned down slightly. The reviewers also raise some other minor comments that should be addressed.}\\
	
	\textbf{We have revised the discussion paragraph on the effect of drift accordingly (ll. 422-437, including new citations by \citealt{Wright1943, Kimura1964, Slatkin1987}), and also a sentence in the results to point out that Eleuthera had a near-significant spatial correlation (ll. 336). We also toned down that statement in the abstract (l. 18).}
	
	\pagebreak
	
	\subsection*{Reviewer 1}
	
	\subsubsection*{Comment 3}
	
	\textit{Authors did a great job at designing this experiment and at combining diverse statistical analyses that enable testing their hypotheses taking into account potential sources of noise such as geographical proximity. I have one major suggestion/comment. Do you know whether such colour variation nicely described by your analyses, is actually detectable by conspecific, congeners and/or predators, the “agents” of selection?  For instance, do you think all dewlaps will be similarly seen under similar light/background conditions? Or whether observers can tell apart dewlaps coming from different habitats? In the first case, it is likely that the non-directional variation that you describe is not under selection, as to observers’ eyes, all dewlap colours look similar. In the second case, the conclusions will be similar to the ones you get with your current approach. This question can be tested applying vision modelling using your reflectance profiles, and an irradiance profile of at least one of the habitats that you studied. Perhaps this has been already done in other studies. If such is the case, you can clearly mention it.}\\
	
	\textbf{A very interesting idea. We do not know how detectable the reported differences in coloration are. Indeed, to know this we would need to do some visual modeling, and for this we would also need some habitat irradiance data, which we do not have unfortunately. Visual modeling has been performed in \textit{Anolis} lizards (e.g. \citealt{Leal2004, Fleishman2020}), but to the best of our knowledge it has not been done on our study populations, and so we do not know if the lizards can detect the differences in coloration of conspecifics from different habitats (now discussed on ll. 450-454).}
	
	\subsubsection*{Comment 4}
	
	\textit{Other remarks:
	-Among the differences that can be found on the three studied habitats, authors mostly talk about differences in light incidence. Are there any differences also in background colour? Predator community? Anole community? If so, how will they contribute or not to the obtained results?}\\
	
	\textbf{Interesting idea. Background color definitely varies between habitats, which can be seen with the naked eye from photos, but we did not quantify this precisely (we do not have irradiance profiles for the habitats telling us which wavelengths are different). Anole and predator communities have been shown to vary between islands (at least in density of the main species, which are overall present across all islands, or at least their ecologically similar species, see e.g. \citealt{Baeckens2018}), but we do not know whether densities also vary between the habitats within islands. We discuss that in a newly added paragraph (ll. 456-467).}
	
	\subsubsection*{Comment 5}
	
	\textit{-How big is the home range of an Anolis? Sorry of I missed it. This information will help understanding the importance of your experimental design.}\\
	
	\textbf{The best estimate of \textit{Anolis sagrei} home ranges \citep{Kamath2018} found them to be 225m$^2$ for males and 36m$^2$ for females, but with overlap (and multiple matings for both sexes, a. k. a. polygynandry). Further, space use and movement were dynamic and varied, which increased both overlap among individuals and multiple mating. But these estimates were measured in an introduced population living on a college campus in Florida and may not represent the real territory size on our study islands. 
	Many anoles are considered territorial owing to their defensive displays, but genetic and reproductive data do not bear out the idea that anoles stay put and only mate with individuals in their territories (\citealt{Kamath2017a, Kamath2018a}, added to the manuscript). Indeed many are polygynandrous. Hence, although the traditional literature suggests that the species is territorial, its ability for movement and promiscuity, its continuous distribution at high densities throughout the islands, and the polyphyly of individuals from different habitats within islands suggests that short and long-range gene flow are highly plausible. Besides, we note that \citet{Kamath2018} estimated movement over a few months time, while these ranges could be substantially larger over a lifetime. In another trunk-ground anole, \textit{Anolis gundlachi}, makes and females moved 80 meters from their point of capture were able to return to the original location 80\% and 40-60\% of the time, respectively, in less than 24 hours (\citealt{Steinberg2017}, added to the manuscript), suggesting at least the capacity to move very long distances in a lizard lifetime. We have added some information to this effect in the introduction (ll. 112-114) and the discussion (ll. 363-371).} 
	
	\subsubsection*{Comment 6}
	
	\textit{-In line 91, what do you mean by climatic variables? Light incidence?}\\
	
	\textbf{We have now added that these climatic variables include precipitation and solar incidence, among others, and those were interpreted as causing differences in types of habitat, with different light environments, among islands (ll. 92-94).}
	
	\subsubsection*{Comment 7}
	
	\textit{-In line 413, perhaps is worth it including a short definition of runaway sexual selection.}\\
	
	\textbf{This is now done (l. 443).}
	
	\pagebreak
	
	\subsection*{Reviewer 2}
	
	\subsubsection*{Comment 8}
	
	\textit{I have only one minor suggestion/comment. The study focuses on the colouration of the dewlap. Of course, also the size and use of the dewlap is essential during signalling (see e.g. papers by Lailvaux, Irschick, Vanhooydonck, Driessens, Husak, Baeckens, to name a few). While I understand this is not the goal of the study, I think it’s still important to invest a couple of sentences or a short paragraph on how other characteristics of the dewlap (aside from colouration) are important too, and that those characteristics (possibly interaction with colouration) might vary within- and among-islands as well.}\\
	
	\textbf{Thank you for your review and for the suggestion. We have added a paragraph to the discussion (ll. 456-467) mentioning those other facets of anole display behavior, and added new references to support them \citep{Vanhooydonck2005, Lailvaux2006, Lailvaux2007, Vanhooydonck2009, Driessens2014, Driessens2015, Lailvaux2015, Driessens2017, Baeckens2018}. (Note that \citealt{Lailvaux2006} has been added to the paragraph that comes just before, on ll. 445-446.)}
	
	\pagebreak
	
	\bibliographystyle{apalike}
	\bibliography{library.bib}
		
\end{document}